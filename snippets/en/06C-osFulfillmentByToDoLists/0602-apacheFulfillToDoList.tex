% Telekom osCompendium 'for being included' snippet template
%
% (c) Karsten Reincke, Deutsche Telekom AG, Darmstadt 2011
%
% This LaTeX-File is licensed under the Creative Commons Attribution-ShareAlike
% 3.0 Germany License (http://creativecommons.org/licenses/by-sa/3.0/de/): Feel
% free 'to share (to copy, distribute and transmit)' or 'to remix (to adapt)'
% it, if you '... distribute the resulting work under the same or similar
% license to this one' and if you respect how 'you must attribute the work in
% the manner specified by the author ...':
%
% In an internet based reuse please link the reused parts to www.telekom.com and
% mention the original authors and Deutsche Telekom AG in a suitable manner. In
% a paper-like reuse please insert a short hint to www.telekom.com and to the
% original authors and Deutsche Telekom AG into your preface. For normal
% quotations please use the scientific standard to cite.
%
% [ Framework derived from 'mind your Scholar Research Framework' 
%   mycsrf (c) K. Reincke 2012 CC BY 3.0  http://mycsrf.fodina.de/ ]
%


%% use all entries of the bibliography
%\nocite{*}

\section{Apache licensed software}

Today, the current release of the Apache open source license is version 2.0,
elder versions are deprecated\footnote{For details $\rightarrow$ OSLiC, pp.\
\pageref{sec:ProtectPowerOfApL}}. Because it focusses primarily on the
\enquote{redistribution}\footcite[cf.][\nopage wp.\ §4]{Apl20OsiLicense2004a},
the following simplified Apache specific open source use case
finder\footnote{For details of the general OSUC finder $\rightarrow$ OSLiC, pp.\
\pageref{OsucTokens} and \pageref{OsucDefinitionTree}} can be used:
 
\tikzstyle{nodv} = [font=\small, ellipse, draw, fill=gray!10, 
    text width=2cm, text centered, minimum height=2em]

\tikzstyle{nods} = [font=\footnotesize, rectangle, draw, fill=gray!20, 
    text width=1.2cm, text centered, rounded corners, minimum height=3em]

\tikzstyle{nodb} = [font=\footnotesize, rectangle, draw, fill=gray!20, 
    text width=2.2cm, text centered, rounded corners, minimum height=3em]
    
\tikzstyle{leaf} = [font=\tiny, rectangle, draw, fill=gray!30, 
    text width=1.2cm, text centered, minimum height=6em]

\tikzstyle{edge} = [draw, -latex']

\begin{tikzpicture}[]

\node[nodv] (l71) at (4,10) {ApL};

\node[nodb] (l61) at (0,8.6) {\textit{recipient:} \\ \textbf{4yourself}};
\node[nodb] (l62) at (6.5,8.6) {\textit{recipient:} \\ \textbf{2others}};

\node[nodb] (l51) at (2.5,7) {\textit{state:} \\ \textbf{unmodified}};
\node[nodb] (l52) at (9.3,7) {\textit{state:} \\ \textbf{modified}};

\node[nods] (l41) at (1.8,5.4) {\textit{form:} \textbf{source}};
\node[nods] (l42) at (3.6,5.4) {\textit{form:} \textbf{binary}};
\node[nodb] (l43) at (6.5,5.4) {\textit{type:} \\ \textbf{proapse}};
\node[nodb] (l44) at (12,5.4) {\textit{type:} \\ \textbf{snimoli}};


\node[nods] (l31) at (5.4,3.8) {\textit{form:} \textbf{source}};
\node[nods] (l32) at (7.2,3.8) {\textit{form:} \textbf{binary}};
\node[nodb] (l33) at (10,3.8) {\textit{context:} \\ \textbf{independent}};
\node[nodb] (l34) at (13.5,3.8) {\textit{context:} \\ \textbf{embedded}};

\node[nods] (l21) at (9,2.2) {\textit{form:} \textbf{source}};
\node[nods] (l22) at (10.8,2.2) {\textit{form:} \textbf{binary}};
\node[nods] (l23) at (12.6,2.2) {\textit{form:} \textbf{source}};
\node[nods] (l24) at (14.4,2.2) {\textit{form:} \textbf{binary}};

\node[leaf] (l11) at (0,0) {\textbf{ApL-C1} \textit{using software only
for yourself}};

\node[leaf] (l12) at (1.8,0) { \textbf{ApL-C2} \textit{ distributing unmodified
software as sources}};

\node[leaf] (l13) at (3.6,0) { \textbf{ApL-C3}  \textit{ distributing unmodified
software as binaries}};

\node[leaf] (l14) at (5.4,0) { \textbf{ApL-C4}  \textit{ distributing modified
program as sources}};

\node[leaf] (l15) at (7.2,0) { \textbf{ApL-C5}  \textit{ distributing modified
program as binaries}};

\node[leaf] (l16) at (9,0) { \textbf{ApL-C6}  \textit{ distributing modified
library as independent sources}};

\node[leaf] (l17) at (10.8,0) { \textbf{ApL-C7} \textit{distributing modified
library as independent binaries}};

\node[leaf] (l18) at (12.6,0) { \textbf{ApL-C8}  \textit{distributing
modified library as embedded sources}};

\node[leaf] (l19) at (14.4,0) { \textbf{ApL-C9}  \textit{ distributing modified
library as embedded binaries}};


\path [edge] (l71) -- (l61);
\path [edge] (l71) -- (l62);
\path [edge] (l61) -- (l11);
\path [edge] (l62) -- (l51);
\path [edge] (l62) -- (l52);
\path [edge] (l51) -- (l41);
\path [edge] (l51) -- (l42);
\path [edge] (l52) -- (l43);
\path [edge] (l52) -- (l44);
\path [edge] (l41) -- (l12);
\path [edge] (l42) -- (l13);
\path [edge] (l43) -- (l31);
\path [edge] (l43) -- (l32);
\path [edge] (l44) -- (l33);
\path [edge] (l44) -- (l34);
\path [edge] (l31) -- (l14);
\path [edge] (l32) -- (l15);
\path [edge] (l33) -- (l21);
\path [edge] (l33) -- (l22);
\path [edge] (l34) -- (l23);
\path [edge] (l34) -- (l24);
\path [edge] (l21) -- (l16);
\path [edge] (l22) -- (l17);
\path [edge] (l23) -- (l18);
\path [edge] (l24) -- (l19);

\end{tikzpicture}


\subsection{ApL-C1: Using the software only for yourself}
\label{OSUC-01-Apache20} \label{OSUC-03-Apache20} 
\label{OSUC-06-Apache20} \label{OSUC-09-Apache20}

\begin{description}

\item[means] that you are going to use a received Apache licensed software only
for yourself and that you do not hand it over to any 3rd party in any sense.

\item[covers] OSUC-01, OSUC-03, OSUC-06, and OSUC-09\footnote{For details 
$\rightarrow$ OSLiC, pp.\ \pageref{OSUC-01-DEF} - \pageref{OSUC-09-DEF}}

\item[requires] no tasks in order to fulfill the conditions of the Apache 2.0
license with respect to this use case:
  \begin{itemize}
    \item You are allowed to use any kind of Apache software in any sense and in
    any context without being obliged to do anything as long as you do not
    give the software to 3rd parties.
  \end{itemize}
  
\item[prohibits] \ldots
\begin{itemize}
  \item to promote any of your services – based on the this software – by
  trademarks, service marks, or product names linked to the software except as
  required for unpartially describing the used software file.
  \item to institute any patent litigation against anyone alleging that the
  software constitutes patent infringement.
\end{itemize}

\end{description}

\subsection{ApL-C2: Passing the unmodified software as source code}
\label{OSUC-02S-Apache20} \label{OSUC-05S-Apache20} \label{OSUC-07S-Apache20} 

\begin{description}

\item[means] that you are going to distribute an unmodified version of the
received Apache software to 3rd parties -- in the form of source code files or
as a source code package. In this case it is not discriminating 
to distribute a program, an application, a server, a snippet, a module, a
library, or a plugin as an independent or as an embedded unit.

\item[covers] OSUC-02S, OSUC-05S, OSUC-07S\footnote{For details $\rightarrow$
OSLiC, pp.\ \pageref{OSUC-02S-DEF} - \pageref{OSUC-07S-DEF}}

\item[requires] the following tasks in order to fulfill the license conditions:
\begin{itemize}
  \item \textbf{[mandatory:]} Give the recipient a copy of the Apache 2.0
  license. If it is not already part of the software package, add
  it\footnote{For implementing the handover of files correctly $\rightarrow$
  OSLiC, p. \pageref{DistributingFilesHint}}.
  \item \textbf{[mandatory:]} Ensure that the licensing elements -- esp.\ the
  specific copyright notice of the original author(s) -- are retained in your
  package in the form you have received them.
  \item \textbf{[mandatory:]} Ensure that a \emph{notice text file}\footnote{
  The Apache license seems purposely to be a bit ambiguous: it uses the term
  \enquote{``Notice'' text file}. In its strict sense, the term refers to a file
  named 'NOTICE.[txt\textbar{}pdf\textbar{}\ldots]'. In a weaker sense, it may
  denote any (text) file containing (licensing) notices. For being sure to act
  according to this requirement you should also read this term in the broader
  sense if there is no text file named 'NOTICE'} is retained in your package in
  the form you have received it.
  
  \item \textbf{[voluntary:]} Let the documentation of your distribution and/or
  your additional material also reproduce the content of the \emph{notice text
  file}, a hint to the software name, a link to its homepage, and a link to the
  Apache 2.0 license.
\end{itemize}

\item[prohibits] \ldots
\begin{itemize}
  \item to promote any of your services or products – based on the this software
  – by trademarks, service marks, or product names linked to this Apache
  software, except as required for unpartially describing the used software and
  for reproducing the notice text file.
  \item to institute any patent litigation against anyone alleging that the
  software constitutes patent infringement.
\end{itemize}

\end{description}


\subsection{ApL-C3: Passing the unmodified software as binaries} 
\label{OSUC-02B-Apache20} \label{OSUC-05B-Apache20} \label{OSUC-07B-Apache20}

\begin{description}
\item[means] that you are going to distribute an unmodified version of the
received Apache software to 3rd parties -- in the form of binary files or as a
bi\-na\-ry package. In this case it is not discriminating to distribute a
program, an application, a server, a snippet, a module, a library,
or a plugin as an independent or an embedded unit.

\item[covers] OSUC-02B, OSUC-05B, OSUC-07B\footnote{For details $\rightarrow$
OSLiC, pp.\ \pageref{OSUC-02B-DEF} - \pageref{OSUC-07B-DEF}}

\item[requires] the following tasks in order to fulfill the license conditions:
\begin{itemize}
  \item \textbf{[mandatory:]} Give the recipient a copy of the Apache 2.0
  license. If it is not already part of the binary package, add
  it\footnote{For implementing the handover of files correctly $\rightarrow$
  OSLiC, p. \pageref{DistributingFilesHint}}.
  
  \item \textbf{[mandatory:]} Ensure that the licensing elements -- esp.\ the
  specific copyright notice of the original author(s) -- are retained in your
  package in the form you have received them. If you compile the binary from the
  sources, ensure that all the licensing elements are also incorporated into the
  package.
  \item \textbf{[mandatory:]} Ensure that the \emph{notice text file} is
  retained or integrated into your binary package in the form you have initially
  received
  it.
  \item \textbf{[mandatory:]} Ensure that the \emph{notice text file} is also
  reproduced if and whereever such third-party notices normally appear --
  especially, if you are distributing an unmodified Apache licensed library as
  embedded component of your own work which displays its own copyright notice.
  
  \item \textbf{[voluntary:]} Let the documentation of your distribution and/or
  your additional material also reproduce the content of the \emph{notice text
  file}, a hint to the software name, a link to its homepage, and a link to the
  Apache 2.0 license -- especially as subsection of your own copyright notice.
\end{itemize}

\item[prohibits] \ldots
\begin{itemize}
  \item to promote any of your services or products – based on the this software
  – by trademarks, service marks, or product names linked to this Apache
  software, except as required for unpartially describing the used software and
  for reproducing the notice text file.
  \item to institute any patent litigation against anyone alleging that the
  software constitutes patent infringement.
\end{itemize}

\end{description}

\subsection{ApL-C4: Passing a modified program as source code}
\label{OSUC-04S-Apache20} 

\begin{description}
\item[means] that you are going to distribute a modified version of the received
Apache licensed program, application, or server (proapse) to 3rd parties -- in
the form of source code files or as a source code package.
\item[covers] OSUC-04S\footnote{For details $\rightarrow$ OSLiC, pp.\
\pageref{OSUC-04S-DEF}}
\item[requires] the tasks in order to fulfill the license conditions:
\begin{itemize}
  
  \item \textbf{[mandatory:]} Give the recipient a copy of the Apache 2.0
  license. If it is not already part of the software package, add
  it\footnote{For implementing the handover of files correctly $\rightarrow$
  OSLiC, p. \pageref{DistributingFilesHint}}.

  \item \textbf{[mandatory:]} Ensure that the licensing elements -- esp.\ the
  specific copyright notice of the original author(s) -- are retained in your
  package in the form you have received them.
  
  \item \textbf{[mandatory:]} Ensure that the \emph{notice text file} contains
  at least all the information of that \emph{notice text file} you have
  received.

  \item \textbf{[mandatory:]} Ensure that the \emph{notice text file} is also
  reproduced if and whereever such third-party notices normally appear. If the
  program already displays a copyright dialog, update it in an appropriate
  manner.
  
  \item \textbf{[mandatory:]} Inside of the source code, mark all your
  modifications thoroughly. Generate a \emph{notice text file}, if it still does
  not exist. \emph{Add} a description of your modifications into the
  \emph{notice text file}.
   
  \item \textbf{[voluntary:]} Let the documentation of your distribution and/or
  your additional material also reproduce the content of the \emph{notice text
  file}, a hint to the software name, a link to its homepage, and a link to the
  Apache 2.0 license.
  
 \end{itemize}
 
\item[prohibits] \ldots
\begin{itemize}
  \item to promote any of your services or products – based on the this software
  – by trademarks, service marks, or product names linked to this Apache
  software, except as required for unpartially describing the used software and
  for reproducing the notice text file.
  \item to institute any patent litigation against anyone alleging that the
  software constitutes patent infringement.
\end{itemize}

\end{description}

\subsection{ApL-C5: Passing a modified program as binary}
\label{OSUC-04B-Apache20}
\begin{description}
\item[means] that you are going to distribute a modified version of the received
Apache licensed pro\-gram, application, or server (proapse) to 3rd parties -- in
the form of binary files or as a binary package.
\item[covers] OSUC-04B\footnote{For details $\rightarrow$ OSLiC, pp.\
\pageref{OSUC-04B-DEF}}
\item[requires] the tasks in order to fulfill the license conditions:
\begin{itemize}

 \item \textbf{[mandatory:]} Give the recipient a copy of the Apache 2.0
  license. If it is not already part of the binary package, add
  it\footnote{For implementing the handover of files correctly $\rightarrow$
  OSLiC, p. \pageref{DistributingFilesHint}}.
  
  \item \textbf{[mandatory:]} Ensure that the licensing elements -- esp.\ the
  specific copyright notice of the original author(s) -- are retained in your
  package in the form you have received them. If you compile the binary from the
  sources, ensure that all the licensing elements are also incorporated into the
  package.
  
  \item \textbf{[mandatory:]} Ensure that the \emph{notice text file} contains
  at least all the information of that \emph{notice text file} you have
  received. If it still does not exist, create it. \emph{Expand} the
  \emph{notice text file} by a description of your modifications.
  
  \item \textbf{[mandatory:]} Ensure that the \emph{notice text file} is also
  reproduced if and whereever such third-party notices normally appear. If the
  program already displays a copyright dialog, update it in an appropriate
  manner.
 
  \item \textbf{[voluntary:]} Even if you do not want to distribute your
  modified source code, mark all your modifications thoroughly.
 
  \item \textbf{[voluntary:]} Let the documentation of your distribution and/or
  your additional material also reproduce the content of the \emph{notice text
  file}, a hint to the software name, a link to its homepage, and a link to the
  Apache 2.0 license -- especially as a subsection of your own copyright notice.

\end{itemize}

\item[prohibits] \ldots
\begin{itemize}
  \item to promote any of your services or products – based on the this software
  – by trademarks, service marks, or product names linked to this Apache
  software, except as required for unpartially describing the used software and
  for reproducing the notice text file.
  \item to institute any patent litigation against anyone alleging that the
  software constitutes patent infringement.
\end{itemize}

\end{description}

\subsection{ApL-C6: Passing a modified library as independent source code}
\label{OSUC-08S-Apache20}

\begin{description}
\item[means] that you are going to distribute a modified version of the received
Apache licensed code snippet, module, library, or plugin (snimoli) to 3rd
parties -- in the form of source code files or as a source code package, but
without embedding it into another larger software unit.
\item[covers] OSUC-08S\footnote{For details $\rightarrow$ OSLiC, pp.\
\pageref{OSUC-08S-DEF}}
\item[requires] the tasks in order to fulfill the license conditions:
\begin{itemize}
  
  \item \textbf{[mandatory:]} Give the recipient a copy of the Apache 2.0
  license. If it is not already part of the software package, add
  it\footnote{For implementing the handover of files correctly $\rightarrow$
  OSLiC, p. \pageref{DistributingFilesHint}}.

  \item \textbf{[mandatory:]} Ensure that the licensing elements -- esp.\ the
  specific copyright notice of the original author(s) -- are retained in your
  package in the form you have received them.
  
  \item \textbf{[mandatory:]} Ensure that the \emph{notice text file} contains
  at least all the information of that \emph{notice text file} you have
  received.
 
  \item \textbf{[voluntary:]} Inside of the source code, mark all your
  modifications thoroughly. Generate a \emph{notice text file}, if it still does
  not exist. \emph{Expand} the \emph{notice text file} by a description of your
  modifications.
   
  \item \textbf{[voluntary:]} Let the documentation of your distribution and/or
  your additional material also reproduce the content of the \emph{notice text
  file}, a hint to the software name, a link to its homepage, and a link to the
  Apache 2.0 license.

\end{itemize}

\item[prohibits] \ldots
\begin{itemize}
  \item to promote any of your services or products – based on the this software
  – by trademarks, service marks, or product names linked to this Apache
  software, except as required for unpartially describing the used software and
  for reproducing the notice text file.
  \item to institute any patent litigation against anyone alleging that the
  software constitutes patent infringement.
\end{itemize}

\end{description}


\subsection{ApL-C7: Passing a modified library as independent binary}
\label{OSUC-08B-Apache20}
\begin{description}
\item[means] that you are going to distribute a modified version of the received
Apache licensed code snippet, module, library, or plugin (snimoli) to 3rd
parties -- in the form of binary files or as a binary package but without
embedding it into another larger software unit.
\item[covers] OSUC-08B\footnote{For details $\rightarrow$ OSLiC, pp.\
\pageref{OSUC-08B-DEF}}
\item[requires] the tasks in order to fulfill the license conditions:
\begin{itemize}
  
 \item \textbf{[mandatory:]} Give the recipient a copy of the Apache 2.0
  license. If it is not already part of the binary package, add
  it\footnote{For implementing the handover of files correctly $\rightarrow$
  OSLiC, p. \pageref{DistributingFilesHint}}.
  
  \item \textbf{[mandatory:]} Ensure that the licensing elements -- esp.\ the
  specific copyright notice of the original author(s) -- are retained in your
  package in the form you have received them. If you compile the binary from the
  sources, ensure that all the licensing elements are also incorporated into the
  package.
  
  \item \textbf{[mandatory:]} Ensure that the \emph{notice text file} contains
  at least all the information of that \emph{notice text file} you have
  received. If it still does not exist, create it. \emph{Expand} the
  \emph{notice text file} by a description of your modifications.
   
  \item \textbf{[voluntary:]} Even if you do not want to distribute your
  modified source code, mark all your modifications thoroughly.
 
  \item \textbf{[voluntary:]} Let the documentation of your distribution and/or
  your additional material also reproduce the content of the \emph{notice text
  file}, a hint to the software name, a link to its homepage, and a link to the
  Apache 2.0 license -- especially as a subsection of your own copyright notice.
  
\end{itemize}

\item[prohibits] \ldots
\begin{itemize}
  \item to promote any of your services or products – based on the this software
  – by trademarks, service marks, or product names linked to this Apache
  software, except as required for unpartially describing the used software and
  for reproducing the notice text file.
  \item to institute any patent litigation against anyone alleging that the
  software constitutes patent infringement.
\end{itemize}

\end{description}

\subsection{ApL-C8: Passing a modified library as embedded source code}
\label{OSUC-10S-Apache20}

\begin{description}
\item[means] that you are going to distribute a modified version of the received
Apache licensed code snippet, module, library, or plugin (snimoli) to 3rd
parties -- in the form of source code files or as a source code package together
with another larger software unit which contains this code snippet, module,
library, or plugin as an embedded component.
\item[covers] OSUC-10S\footnote{For details $\rightarrow$ OSLiC, pp.\
\pageref{OSUC-10S-DEF}}
\item[requires] the tasks in order to fulfill the license conditions:
\begin{itemize}
  
  \item \textbf{[mandatory:]} Give the recipient a copy of the Apache 2.0
  license. If it is not already part of the software package, add
  it\footnote{For implementing the handover of files correctly $\rightarrow$
  OSLiC, p. \pageref{DistributingFilesHint}}.

  \item \textbf{[mandatory:]} Ensure that the licensing elements -- esp.\ the
  specific copyright notice of the original author(s) -- are retained in your
  package in the form you have received them.
  
  \item \textbf{[mandatory:]} Ensure that the \emph{notice text file} contains at least
  all the information of that \emph{notice text file} you have received.
 
  \item \textbf{[mandatory:]} Ensure that the \emph{notice text file} is also
  reproduced if and whereever such third-party notices normally appear. If your
  overarching program displays its own copyright dialog, insert this information
  there.
 
  \item \textbf{[mandatory:]} Inside of the library\footnote{or snippet, or
  module, or plugin} source code, mark all your modifications thoroughly.
  Generate a \emph{notice text file}, if it still does not exist. \emph{Expand}
  the \emph{notice text file} by a description of your modifications.
  
  \item \textbf{[voluntary:]} Let the documentation of your distribution and/or
  your additional material also reproduce the content of the \emph{notice text
  file}, a hint to the software name, a link to its homepage, and a link to the
  Apache 2.0 license.

  \item \textbf{[voluntary:]} Arrange your source code distribution so that the
  integrated Apache license and the \emph{notice text file} clearly refer only
  to the embedded library and do not disturb the licensing of your own
  overarching work. It's a good tradition to keep the embedded components like
  libraries, modules, snippets, or plugins in specific directory which contains
  also all additional licensing elements.
 
\end{itemize}

\item[prohibits] \ldots
\begin{itemize}
  \item to promote any of your services or products – based on the this software
  – by trademarks, service marks, or product names linked to this Apache
  software, except as required for unpartially describing the used software and
  for reproducing the notice text file.
  \item to institute any patent litigation against anyone alleging that the
  software constitutes patent infringement.
\end{itemize}

\end{description}


\subsection{ApL-C9: Passing a modified library as embedded binary}
\label{OSUC-10B-Apache20}
\begin{description}
\item[means] that you are going to distribute a modified version of the received
Apache licensed code snippet, module, library, or plugin to 3rd parties -- in
the form of binary files or as a binary package together with another larger
software unit which contains this code snippet, module, library, or plugin as an
embedded component.
\item[covers] OSUC-10B\footnote{For details $\rightarrow$ OSLiC, pp.\
\pageref{OSUC-10B-DEF}}
\item[requires] the tasks in order to fulfill the license conditions:
\begin{itemize}
  
  \item \textbf{[mandatory:]} Give the recipient a copy of the Apache 2.0
  license. If it is not already part of the binary package, add
  it\footnote{For implementing the handover of files correctly $\rightarrow$
  OSLiC, p. \pageref{DistributingFilesHint}}.
  
  \item \textbf{[mandatory:]} Ensure that the licensing elements -- esp.\ the
  specific copyright notice of the original author(s) -- are retained in your
  package in the form you have received them. If you compile the binary from the
  sources, ensure that all the licensing elements are also incorporated into the
  package.
  
  \item \textbf{[mandatory:]} Ensure that the \emph{notice text file} contains
  at least all the information of that \emph{notice text file} you have
  received.  If it still does not exist, create it. \emph{Expand} the
  \emph{notice text file} by a description of your modifications.
 
  \item \textbf{[mandatory:]} Ensure that the \emph{notice text file} is also
  reproduced if and whereever such third-party notices normally appear. If your
  overarching program displays its own copyright dialog, insert this information
  there.
     
  \item \textbf{[voluntary:]} Even if you do not want to distribute your
  modified source code, mark all your modifications of the embedded
  libary\footnote{or snippet, or module, or plugin} thoroughly.
 
  \item \textbf{[voluntary:]} Let the documentation of your distribution and/or
  your additional material also reproduce the content of the \emph{notice text
  file}, a hint to the software name, a link to its homepage, and a link to the
  Apache 2.0 license -- especially as subsection of your own copyright notice.
  
 \item \textbf{[voluntary:]} Arrange your binary distribution so that the
  integrated Apache license and the \emph{notice text file} clearly refer only
  to the embedded library and do not disturb the licensing of your own
  overarching work. It's a good tradition to keep the libraries, modules,
  snippet, or plugins in specific directories which contain also all licensing
  elements.
  
\end{itemize}

\item[prohibits] \ldots
\begin{itemize}
  \item to promote any of your services or products – based on the this software
  – by trademarks, service marks, or product names linked to this Apache
  software, except as required for unpartially describing the used software and
  for reproducing the notice text file.
  \item to institute any patent litigation against anyone alleging that the
  software constitutes patent infringement.
\end{itemize}

\end{description}

\subsection{Discussions and Explanations}
\begin{itemize}
  \item On the one hand, the Apache 2.0 license does not permit
  \enquote{[\ldots] to use the trade names, trademarks, service marks, or
  product names of the Licensor, except as required for reasonable and customary
  use in describing the origin of the Work and reproducing the content of the
  NOTICE file}\footcite[cf.][\nopage wp.\ §6]{Apl20OsiLicense2004a}. On the other
  hand, this license alerts that all the patent licenses – granted to those who
  \enquote{[\ldots] institute a patent litigation} – will terminate
  automatically\footcite[cf.][\nopage wp.\ §3]{Apl20OsiLicense2004a}. Hence, the
  OSLiC generally (ApL-C1 - ApL-C9) interdicts to promote products or services by
  these elements and to legally fight against patents linked to the software.
  
  \item The ApL also requires to \enquote{[\ldots] give any other recipients of
  the Work or Derivative Works a copy of this License}\footcite[cf.][\nopage wp.\
  §4.1]{Apl20OsiLicense2004a}. Therefore, all \emph{2others} use cases contain
  the respective mandatory condition (ApL-C2 - ApL-C9).
   
  \item Additionally, the ApL requires, that modifications must be
  marked\footcite[cf.][\nopage wp.\ §4.2]{Apl20OsiLicense2004a}. Thus, in all
  cases of passing the modified software in the form of source code the OSLiC
  requires to mark the modifications and to integrate a hint into the notice
  file – while in all the cases of passing the modified software in the form of
  binaries it inserts only a voluntary condition (ApL-C4 - ApL-C9).
  
  \item Furthermore, the ApL requires that one must \enquote{[\ldots] retain, in
  the Source form of any Derivative Works that You distribute, all copyright,
  patent, trademark, and attribution notices from the Source form of the Work}
  So, the OSLIC requires in all contexts (ApL-C1 - ApL-C9) that the licensing
  elements are retained in the form you have received them\footnote{This might
  confuse some readers: Yes, even if you distribute a modified version in the
  form of binaries you must fulfill this condition. Moreover, you must also hand
  the license over to your receipient. But, nevertheless, you are not obliged to
  publish the modified source code, too. ($\rightarrow$ OSLiC, p.
  \pageref{sec:ProtectPowerOfApL})}.
  
  \item Finally, the ApL requires that the received “NOTICE text file” must be
  integrated as readable copy to each package distributed in the form of source
  code, or – in case of binary distibutions – must be displayed
  \enquote{[\ldots] if and wherever such third-party notices normally
  appear}\footcite[cf.][\nopage wp.\ §4.4]{Apl20OsiLicense2004a}. Thus, the OSLiC
  requires mandatorily that all source code distributions must include the
  notice text file (ApL-C2, ApL-C4, ApL-C6, ApL-C8) and that all distributions of
  binary applications which normally show such a copyrigth screen must integrate
  the content of the notice file into this screen (ApL-C5, ApL9). For libraries
  distributed in the form of binaries it is assumed that they normally do not
  contain such copyright dialogs (ApL-C7)
\end{itemize}








%\bibliography{../../../bibfiles/oscResourcesEn}
