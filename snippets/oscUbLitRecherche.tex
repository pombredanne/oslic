%Telekom osCompendium uni library search explanation
%
% (c) Karsten Reincke, Deutsche Telekom AG, Darmstadt 2011
%
% This LaTeX-File is licensed under the Creative Commons Attribution-ShareAlike
% 3.0 Germany License (http://creativecommons.org/licenses/by-sa/3.0/de/): Feel
% free 'to share (to copy, distribute and transmit)' or 'to remix (to adapt)'
% it, if you '... distribute the resulting work under the same or similar
% license to this one' and if you respect how 'you must attribute the work in
% the manner specified by the author ...':
%
% In an internet based reuse please link the reused parts to www.telekom.com and
% mention the original authors and Deutsche Telekom AG in a suitable manner. In
% a paper-like reuse please insert a short hint to www.telekom.com and to the
% original authors and Deutsche Telekom AG into your preface. For normal
% quotations please use the scientific standard to cite.
%
% [ File structure derived from 'mind your Scholar Research Framework' 
%   mycsrf (c) K. Reincke CC BY 3.0  http://mycsrf.fodina.de/ ]

%
\documentclass[DIV=calc,BCOR=5mm,11pt,headings=small,oneside,abstract=false, toc=bib]{scrartcl}

%%% (1) general configurations %%%
\usepackage[utf8]{inputenc}

%%% (2) language specific configurations %%%
\usepackage[]{a4,ngerman}
\usepackage[english,ngerman]{babel}
\selectlanguage{ngerman}

%language specific quoting signs
%default for language emglish is american style of quotes
\usepackage[german=quotes]{csquotes}

% jurabib configuration
\usepackage[see]{jurabib}
\bibliographystyle{jurabib}
% Telekom osCompendium German Jurabib Configuration Include Module 
%
% (c) Karsten Reincke, Deutsche Telekom AG, Darmstadt 2011
%
% This LaTeX-File is licensed under the Creative Commons Attribution-ShareAlike
% 3.0 Germany License (http://creativecommons.org/licenses/by-sa/3.0/de/): Feel
% free 'to share (to copy, distribute and transmit)' or 'to remix (to adapt)'
% it, if you '... distribute the resulting work under the same or similar
% license to this one' and if you respect how 'you must attribute the work in
% the manner specified by the author ...':
%
% In an internet based reuse please link the reused parts to www.telekom.com and
% mention the original authors and Deutsche Telekom AG in a suitable manner. In
% a paper-like reuse please insert a short hint to www.telekom.com and to the
% original authors and Deutsche Telekom AG into your preface. For normal
% quotations please use the scientific standard to cite.
%
% [ File structure derived from 'mind your Scholar Research Framework' 
%   mycsrf (c) K. Reincke CC BY 3.0  http://mycsrf.fodina.de/ ]

% the first time cite with all data, later with shorttitle
\jurabibsetup{citefull=first}

%%% (1) author / editor list configuration
%\jurabibsetup{authorformat=and} % uses 'und' instead of 'u.'
% therefore define your own abbreviated conjunction: 
% an 'and before last author explicetly written conjunction

% for authors in citations
\renewcommand*{\jbbtasep}{ u. } % bta = between two authors sep
\renewcommand*{\jbbfsasep}{, } % bfsa = between first and second author sep
\renewcommand*{\jbbstasep}{ u. }% bsta = between second and third author sep
% for editors in citations
\renewcommand*{\jbbtesep}{ u. } % bta = between two authors sep
\renewcommand*{\jbbfsesep}{, } % bfsa = between first and second author sep
\renewcommand*{\jbbstesep}{ u. }% bsta = between second and third author sep

% for authors in literature list
\renewcommand*{\bibbtasep}{ u. } % bta = between two authors sep
\renewcommand*{\bibbfsasep}{, } % bfsa = between first and second author sep
\renewcommand*{\bibbstasep}{ u. }% bsta = between second and third author sep
% for editors  in literature list
\renewcommand*{\bibbtesep}{ u. } % bte = between two editors sep
\renewcommand*{\bibbfsesep}{, } % bfse = between first and second editor sep
\renewcommand*{\bibbstesep}{ u. }% bste = between second and third editor sep

% use: name, forname, forname lastname u. forname lastname
\jurabibsetup{authorformat=firstnotreversed}
\jurabibsetup{authorformat=italic}

%%% (2) title configuration
% in every case print the title, let it be seperated from the 
% author by a colon and use the slanted font
\jurabibsetup{titleformat={all,colonsep}}
%\renewcommand*{\jbtitlefont}{\textit}

%%% (3) seperators in bib data
% separate bibliographical hints and page hints by a comma
\jurabibsetup{commabeforerest}

%%% (4) specific configuration of bibdata in quotes / footnote
% use a.a.O if possible
\jurabibsetup{ibidem=strict}

% replace ugly a.a.O. by ders., a.a.O. resp. ders., ebda.
% but if there are more than one author or girl writers?
\AddTo\bibsgerman{
  \renewcommand*{\ibidemname}{Ds., a.a.O.}
  \renewcommand*{\ibidemmidname}{ds., a.a.O.}
}
\renewcommand*{\samepageibidemname}{Ds., ebda.}
\renewcommand*{\samepageibidemmidname}{ds., ebda.}

%%% (5) specific configuration of bibdata in bibliography
% ever an in: before journal and collection/book-tiltes 
\renewcommand*{\bibbtsep}{in: }
%\renewcommand*{\bibjtsep}{in: }

% ever a colon after author names 
\renewcommand*{\bibansep}{: }
% ever a semi colon after the title 
\renewcommand*{\bibatsep}{; }
% ever a comma before date/year
\renewcommand*{\bibbdsep}{, }

% let jurabib insert the S. and p. information
% no S. necessary in bib-files and in cites/footcites
\jurabibsetup{pages=format}

% use a compressed literature-list using a small line indent
\jurabibsetup{bibformat=compress}
\setlength{\jbbibhang}{1em}

% which follows the design of the cites and offers comments
\jurabibsetup{biblikecite}

% print annotations into bibliography
\jurabibsetup{annote}
\renewcommand*{\jbannoteformat}[1]{{ \itshape #1 }}

%refine the prefix of url download
\AddTo\bibsgerman{\renewcommand*{\urldatecomment}{Referenzdownload: }}

% we want to have the year of articles in brackets
\renewcommand*{\bibaldelim}{(}
\renewcommand*{\bibardelim}{)}

%Umformatierung des Reihentitels und der Reihennummer
\DeclareRobustCommand{\numberandseries}[2]{%
\unskip\unskip%,
\space\bibsnfont{(=~#2}%
\ifthenelse{\equal{#1}{}}{)}{, [Bd./Nr.]~#1)}%
}%


% language specific hyphenation
% Telekom osCompendium osHyphenation Include Module
%
% (c) Karsten Reincke, Deutsche Telekom AG, Darmstadt 2011
%
% This LaTeX-File is licensed under the Creative Commons Attribution-ShareAlike
% 3.0 Germany License (http://creativecommons.org/licenses/by-sa/3.0/de/): Feel
% free 'to share (to copy, distribute and transmit)' or 'to remix (to adapt)'
% it, if you '... distribute the resulting work under the same or similar
% license to this one' and if you respect how 'you must attribute the work in
% the manner specified by the author ...':
%
% In an internet based reuse please link the reused parts to www.telekom.com and
% mention the original authors and Deutsche Telekom AG in a suitable manner. In
% a paper-like reuse please insert a short hint to www.telekom.com and to the
% original authors and Deutsche Telekom AG into your preface. For normal
% quotations please use the scientific standard to cite.
%
% [ File structure derived from 'mind your Scholar Research Framework' 
%   mycsrf (c) K. Reincke CC BY 3.0  http://mycsrf.fodina.de/ ]
%


\hyphenation{rein-cke}




%%% (3) layout page configuration %%%

% select the visible parts of a page
% S.31: { plain|empty|headings|myheadings }
%\pagestyle{myheadings}
\pagestyle{headings}

% select the wished style of page-numbering
% S.32: { arabic,roman,Roman,alph,Alph }
\pagenumbering{arabic}
\setcounter{page}{1}

% select the wished distances using the general setlength order:
% S.34 { baselineskip| parskip | parindent }
% - general no indent for paragraphs
\setlength{\parindent}{0pt}
\setlength{\parskip}{1.2ex plus 0.2ex minus 0.2ex}


%%% (4) general package activation %%%
%\usepackage{utopia}
%\usepackage{courier}
%\usepackage{avant}
\usepackage[dvips]{epsfig}

% graphic
\usepackage{graphicx,color}
\usepackage{array}
\usepackage{shadow}
\usepackage{fancybox}

%- start(footnote-configuration)
%  flush the cite numbers out of the vertical line and let
%  the footnote text directly start in the left vertical line
\usepackage[marginal]{footmisc}
%- end(footnote-configuration)

\usepackage{amssymb}
\setcounter{secnumdepth}{5}
\setcounter{tocdepth}{5}

\begin{document}

%% use all entries of the bliography
\nocite{*}

%%-- start(titlepage)
\titlehead{\textbf{O}pen \textbf{S}ource \textbf{Li}cense \textbf{C}ompendium}
%\subject{Sekundärtliteratur zum Thema {\itshape Open Source}}
%\title{Nutzung der Uni-Bibliotheken DA \& FaM}
\subject{Die Uni-Bibliotheken DA und FaM}
\title{Recherche zum Thema 'Open Source License Management'}
\subtitle{Besonderheiten bei der Literatursuche u. -bestellung}
\author{K. Reincke% Telekom osCompendium License Include Module
%
% (c) Karsten Reincke, Deutsche Telekom AG, Darmstadt 2011
%
% This LaTeX-File is licensed under the Creative Commons Attribution-ShareAlike
% 3.0 Germany License (http://creativecommons.org/licenses/by-sa/3.0/de/): Feel
% free 'to share (to copy, distribute and transmit)' or 'to remix (to adapt)'
% it, if you '... distribute the resulting work under the same or similar
% license to this one' and if you respect how 'you must attribute the work in
% the manner specified by the author ...':
%
% In an internet based reuse please link the reused parts to www.telekom.com and
% mention the original authors and Deutsche Telekom AG in a suitable manner. In
% a paper-like reuse please insert a short hint to www.telekom.com and to the
% original authors and Deutsche Telekom AG into your preface. For normal
% quotations please use the scientific standard to cite.
%
% [ File structure derived from 'mind your Scholar Research Framework' 
%   mycsrf (c) K. Reincke CC BY 3.0  http://mycsrf.fodina.de/ ]
%
\footnote{
This text is licensed under the Creative Commons Attribution-ShareAlike 3.0 Germany
License (http://creativecommons.org/licenses/by-sa/3.0/de/): Feel free \enquote{to
share (to copy, distribute and transmit)} or \enquote{to remix (to
adapt)} it, if you \enquote{[\ldots] distribute the resulting work under the
same or similar license to this one} and if you respect how \enquote{you
must attribute the work in the manner specified by the author(s)
[\ldots]}):
\newline
In an internet based reuse please mention the initial authors in a suitable
manner, name their sponsor \textit{Deutsche Telekom AG} and link it to
\texttt{http://www.telekom.com}. In a paper-like reuse please insert a short
hint to \texttt{http://www.telekom.com}, to the initial authors, and to their
sponsor \textit{Deutsche Telekom AG} into your preface. For normal citations
please use the scientific standard.
\newline
{ \tiny \itshape [derived from myCsrf (= 'mind your Scholar Research Framework') 
\copyright K. Reincke CC BY 3.0  http://mycsrf.fodina.de/)] }}}
\maketitle
%%-- end(titlepage)

\begin{abstract}
\noindent \itshape
Dieses Paper erläutert, wie man an der \emph{Universitäts- und Landesbibliothek
Darmstadt}\footnote{\cite[s.][]{UlbDaHome}} oder der
\emph{Universitätsbibliothek Frankfurt a.M.}\footnote{\cite[s.][]{UbFaMHome}}
Literatur zum Thema \emph{Open Source} findet, resp. deren Bibliotheksstandort
zwecks Ausleihe ermittelt.
\end{abstract}

%% no table of content in a snippet
\footnotesize
\tableofcontents
\normalsize

\section{Werkzentrierte Suche: Standort bekannter Werke ermitteln}

\subsection{Monographien und Lehrbücher}

Ausgangspunkt sind die bekannten bibliographischen Daten eines Werkes. Es gilt
dann, seinen Bibliotheksstandort über die Suche nach Autor, Titel etc. zu
ermitteln und es zur Lektüre \enquote{in den Lesesaal} etc. zu bestellen
resp. die elektronische Volltextversion einzusehen und/oder downzuloaden.

\subsubsection{\ldots in der UB Frankfurt}

\begin{itemize}
  \item Es gibt 3 Einstiegspunkte für die zugrundliegende FaM
  OPAC\footnote{Online Public Access Catalogue}-Suche:
  \begin{itemize}
  \item Gehe zu {\ttfamily http://www.ub.uni-frankfurt.de/} und wähle dort
  im linken Menu \emph{Kataloge} an.
  \item Oder gehe direkt zu {\ttfamily http://suche.ub.uni-frankfurt.de/}.
  \item Oder gehe direkt zum älteren Frontend des
  Frankfurter OPAC Katalog {\ttfamily https://lbsopac.rz.uni-frankfurt.de/} (er
  zeigt in der Trefferliste auch an, ob es sich um eine elektronisches oder ein
  gedrucktes Werk handelt.)
  \end{itemize}
  \item Logge Dich ein mit den Daten Deines Bibliotheksausweises
  \item Wähle - je nach Fokus - rechts oben den UB-Katalog oder rechts mittig
  den ebook-katalog an.
  \item Gib den Autornamen, Kernwörter des Titels oder Herausgebernamen ins
  lineare Suchfeld ein.
  \item Finde das gewünschte Buch in der Trefferliste
\end{itemize}

Im Erfolgsfall kann das gesuchte Werk in der Trefferliste angeklickt werden. Um
es real einsehen zu können, sind vier Fälle zu unterscheiden:
\begin{itemize}
  \item Liegt es gebunden vor und ist nicht vorbestellt, kann es im
  erscheinenden OPAC Formular direkt in die Buchausgabe der UB (ggfls. in den
  Lesesaal) bestellt werden und liegt dann dort (1 Woche) zur
  Einsicht/Ausleihe bereit. [\emph{verifiziert am} {\ttfamily 2011.07.01}]
  \item Liegt es gebunden vor und ist vorbestellt, kann es im erscheinenden OPAC
  Formular direkt vorbestellt werden. Nach der Benachrichtigung kann dan wie
  o.a. verfahren werden.  [\emph{verifiziert am} {\ttfamily 2011.07.08}]
  \item Liegt es als elektronische Resource vor, kann es im UB Netz direkt als
  Volltext eingesehen resp. als PDF auf einen Stick heruntergeladen werden. 
  [\emph{verifiziert am} {\ttfamily 2011.07.08}]
  \item Liegt es gebunden vor, ist aber nicht ausleihbar, so muss es direkt
  unter dem angegebenen Standort im Lesesaal / Freihandbestand eingesehen werden. 
  [\emph{verifiziert am} {\ttfamily 2011.07.01}]
\end{itemize}

Wenn das Werk (von außerhalb des UNI/UB-Netzes) nicht über den Katalog der
UB FaM zu finden ist, gibt es folgende Alternativen:

\begin{itemize}
  \item Gehe ins Gebäude der UB FaM an einen Rechner im UB-Netz, und wiederhole
  obiges Verfahren sicherheitshalber innerhalb des UB/UNI-Netzes. \item Wähle
  insbesondere unter {\ttfamily http://suche.ub.uni-frankfurt.de/} rechts mittig
  die Datenbank ebooks (Books \ldots online) an. Oder nutze den allgemeinen
  Einstieg {\ttfamily http://www.ub.uni-frankfurt.de/}, wähle li. oben den
  Menueintrag \enquote{Datenbanken, E-Journals, E-Books} und recherchiere
  in den elektronischen Büchern.  [\emph{verifiziert am} {\ttfamily 2011.07.08}]
  \item Wähle unter {\ttfamily http://suche.ub.uni-frankfurt.de/} - sofern das
  gesuchte Buch weder in den gedruckten, noch in den elektronischen Werken zu
  finden ist - rechts oben satt des UB Kataloges den HeBiS-Katalog an oder gehe
  direkt zu {\ttfamily https://www.portal.hebis.de}, logge Dich dort mit Deinen
  UB FaM Daten ein und initiiere nach Erfolg ein Fernleihe.  [\emph{verifiziert
  am} {\ttfamily 2011.07.08}]
\end{itemize}

\subsubsection{\ldots in der UB Darmstadt}

\begin{itemize}
  \item Es gibt auch für DA 3 Einstiegspunkte für die (OPAC-)Suche:
  \begin{itemize}
  \item Gehe zu {\ttfamily http://www.ulb.tu-darmstadt.de/} und wähle dort
  in der echte Hälfte den Link \emph{Online-Katalog der ULB (DA)} an.
  \item Oder gehe direkt zu {\ttfamily http://pica1l.lhb.tu-darmstadt.de/}.
  \item Oder gehe direkt zum \emph{Darmstädter Katalog Portal}, auch genannt
  \emph{DAKAPO} {\ttfamily http://dakapo.ulb.tu-darmstadt.de/}
  \end{itemize}
  \item Logge Dich ggfls. mittels Deines Bibliotheksausweises ein
  \item Recherchiere mit den Angaben zu Autoren / Herausgeber / Titel
  \item Finde das gewünschte Buch in der Trefferliste
\end{itemize}

Im Erfolgsfall kann das gesuchte Werk in der Trefferliste angeklickt werden. Um
es einzusehen, sind vier Fälle zu unterscheiden:
\begin{itemize}
  \item Liegt es gebunden vor und ist nicht vorbestellt, kann es im
  erscheinenden Formular direkt in die Buchausgabe der UB (ggfls. in den
  Lesesaal) bestellt werden und liegt dann dort (2 Wochen?) zur
  Einsicht/Ausleihe bereit.  [\emph{verifiziert am} {\ttfamily ????.??.??}]
  \item Liegt es gebunden vor und ist vorbestellt, kann es im erscheinenden OPAC
  Formular direkt vorbestellt werden. Nach der Benachrichtigung kann dan wie
  o.a. verfahren werden.  [\emph{verifiziert am} {\ttfamily ????.??.??}]
  \item Liegt es als elektronische Resource vor, kann es im UB Netz direkt als
  Volltext eingesehen resp. als PDF auf einen Stick heruntergeladen werden. 
  [\emph{verifiziert am} {\ttfamily ????.??.??}]
  \item Liegt es gebunden vor, ist aber nicht ausleihbar, so muss es direkt
  unter dem angegebenen Standort im Lesesaal / Freihandbestand eingesehen werden. 
  [\emph{verifiziert am} {\ttfamily ????.??.??}]
\end{itemize}

Wenn das Werk (von außerhalb des UNI/UB-Netzes) nicht über den Katalog der
UB FaM zu finden ist, gibt es folgende Alternativen:

\begin{itemize}
  \item Gehe ins Gebäude der ULB DA an einen (roten) Rechner im UB-Netz und
  wiederhole obiges Verfahren sicherheitshalber.
  \item Wähle zusätzlich unter {\ttfamily http://www.ulb.tu-darmstadt.de/}
  den Link \emph{ebooks} oder nutze direkt {\ttfamily
  http://ebooks.ulb.tu-darmstadt.de/}. Dort gefundene Bücher könne im ULB-Netz
  als PDF downgeloaded werden.
  \item Wähle im Dakapo unter {\ttfamily http://dakapo.ulb.tu-darmstadt.de/} -
  sofern das gesuchte Buch weder edruckt, noch elektronisch in der ULB DA zu
  finden ist - einen alternative Ausleihort und initiiere eine Fernleihe.
\end{itemize}

\subsection{Zeitschriftenartikel}

Sind die bibliographischen Angaben eines Zeitschriftenartikels bekannt,
\enquote{reduziert} sich die Recherche auf die Ermittlung des Standorts der
Zeitschrift und das Holen (Kopieren) des Artikels. Herausforderung ist hier,
dass wesentlich mehr Zeitschriften nur elektronisch existieren.

\subsubsection{Zeitschriften}

Gedruckte / gebundene Zeitschriften als solche werden wie Bücher behandelt. Für
die Suche nach elektronische Zeitschriften gibt es eine
universitätsübergreifende Bibliothek, die \emph{EZB} [\enquote{Elektronische
Zeitschriftenbibliothek}]. Sie wird von allen Unis verlinkt, allerdings
jeweils unterschiedlich parametrisiert, sodass die von der UB gekauften
Zeitschriften auch über das je spezifische Netz als Voltext eingesehen werden
können\footnote{Grüne Ampel: Volltext frei im Internet, gelbe Ampel: UB
spezifisch, Volltext nur über UB)}
\paragraph{\ldots in der UB Frankfurt}
\begin{itemize}
  \item Verfahre wie bei einem gebundenen gedruckten Buch und suche über den UB
  FaM Katalog nach dem Zeitschriftentitel (nicht dem Titel/Autor des Artikels
  in der Zeitschrift)
  \item  gehe zu {\ttfamily http://www.ub.uni-frankfurt.de/} und dort per
  Menüeintrag \enquote{Datenbanken, E-Journals, E-Books} und den dabei
  erscheinenden Link \enquote{Elektronische Zeitschriftenbibliothek}. Hinter
  diesem Eintrag verbirgt sich der Link auf die EZB, der für UB FaM
  parametrisiert worden ist: {\ttfamily
  http://ezb.uni-regensburg.de/fl.phtml?bibid=UBFM}.
  \item Wähle in der EZB Dein Fachgebiet und in dessen Liste Deine
  Zeitschrift.
\end{itemize}

Wenn die Ampel der gefundenen elektronische Zeitschrift grün ist, ist die
Zeitschrift allgemein im Netz erhältlich. Wenn die Ampel gelb ist, kann sie nur
im Netz der UB eingesehen resp. als PDF im Volltext runtergeladen werden. Und
wenn sie rot ist, kann sie nur über Fernleihe bestellt werden.
  
Insgesamt gibt es im Erfolgsfall drei Möglichkeiten:
\begin{itemize}
  \item Liegt die Zeitschrift nur gebunden vor, muss der entsprechende Band -
  wie ein Buch - zur Ansicht in den Lesesaal bestellt werden.
  \item Liegt die Zeitschrift (auch) elektronisch vor und ist ihre Ampel im
  EZB grün, kann der entsprechende Band übers Internet frei eingesehen und der
  Artikel (ausserhalb des UB-Netzes) als PDF downgeloaded werden.
  \item Liegt die Zeitschrift (auch) elektronisch vor und ist ihre Ampel im
  EZB gelb, kann der entsprechende Band nur über das UB-Netz eingesehen und
  der Artikel als PDF downgeloaded werden.
\end{itemize}  

\paragraph{\ldots in der UB Darmstadt}

\begin{itemize}
  \item Verfahre wie bei einem gebundenen gedruckten Buch und suche über den ULB
  DA Katalog nach dem Zeitschriftentitel (nicht dem Titel/Autor des Artikels
  in der Zeitschrift)
  \item  Gehe zu {\ttfamily http://www.ulb.tu-darmstadt.de/} und wähle auf der
  rechten Seite den Link \enquote{Elektronische Zeitschriften (EZB)} aus.
  Dahinter steht der für die ULB Darmstadt parametrisierte Link zur
  \emph{Elektronischen Zeitschriften Bibliothek} \\{\ttfamily
  http://rzblx1.uni-regensburg.de/ezeit/fl.phtml?bibid=TUDA}.
  \item Wähle in der EZB Dein Fachgebiet und in dessen liste Deine Zeitschrift.
  \item Wenn deren Ampel grün ist, ist die Zeitschrift allgemein im Netz
  erhältlich. Wenn die Ampel gelb ist, kann sie nur im Netz der UB eingesehen
  resp. als PDF im Volltext runtergeladen werden. Und wenn sie rot ist, kann sie
  nur über Fernleihe bestellt werden.
\end{itemize}

Insgesamt gibt es im Erfolgsfall auch hier drei Möglichkeiten:
\begin{itemize}
  \item Liegt die Zeitschrift nur gebunden vor, muss der entsprechende Band -
  wie ein Buch - zur Ansicht in den Lesesaal bestellt werden.
  \item Liegt die Zeitschrift (auch) elektronisch vor und ist ihre Ampel im
  EZB grün, kann der entsprechende Band übers Internet frei eingesehen und der
  Artikel (ausserhalb des UB-Netzes) als PDF downgeloaded werden.
  \item Steht die Ampel im EZB dagegen auf gelb, kann der entsprechende Band nur
  über das UB-Netz (rote Rechner) eingesehen und der Artikel als PDF
  downgeloaded werden.
\end{itemize}  

\subsubsection{Zeitschriftenartikel}
\begin{itemize}
  \item Bei einer gebunden Zeitschrift muss die Zeitschrift in den den Lesensaal
  bestellt werden, um den Artikel kopieren zu können.
  \item Bei elektronischen Artikeln kann er bei grüner oder gelber Ampel
  direkt eingesehen und als PDF downgeloaded werden.
\end{itemize}

\subsection{Sammlungen und Periodica}

In beiden Bibliotheken werden Sammlungen analog zu Büchern und Periodica analog
zu Büchern oder Zeitschriften behandelt. Auch hier ist der Unterschied zwischen
gebunden / gedruckten Werken und elektronischen entsprechend zu beachten.

\section{Themenzentrierte Suche: relevante Literatur finden }

\subsection{Schlagwortsuche im Title, Abstract und den Keywords}
Der eine Weg zur thematische Suche führt über die Schlagwortsuche im Titel, im
Abstract und in den Keywords. Technisch unterscheidet sich diese nicht wirklich
von der Suche nach bekannten bibliographischen Daten im UB-Bestand. In beiden
Fällen auf den OPAC-Katalog zugegriffen. In das Suchfeld werden eben nur
thematische Suchbegriffe eingegeben, wobei ggf. die Detailssuche bemüht wird.
Hier noch einmal die Links auf die Katalogsuche:

\subsubsection{Frankfurt}
\begin{itemize}
  \item books \& ebooks (UB-Katalog): {\ttfamily
  http://suche.ub.uni-frankfurt.de/}
  \item books \& ebooks (UB-Katalog): {\small \ttfamily
  https://lbsopac.rz.uni-frankfurt.de/}
  \item books \& ebooks: {\ttfamily
  https://lbsopac.rz.uni-frankfurt.de/}
  \item ebooks: {\ttfamily \tiny
  http://www.ub.uni-frankfurt.de/datenbanken/ebooks\_gesamt.html}
\end{itemize}
\subsubsection{Darmstadt}
\begin{itemize}
  \item books \& ebooks (UB-Katalog): {\ttfamily
  http://www.ulb.tu-darmstadt.de/}
  \item books \& ebooks (UB-Katalog): {\ttfamily
  http://pica1l.lhb.tu-darmstadt.de/}
  \item books \& ebooks (Dakapo): {\ttfamily http://dakapo.ulb.tu-darmstadt.de/}
  \item ebooks: {\ttfamily http://ebooks.ulb.tu-darmstadt.de/}
\end{itemize}

Gesucht wird in all diesen Fällen auf Metadaten. Das bedeutet konsequenterweise,
dass so nicht nach Zeitschriftenartikeln gesucht werden kann. Das wäre erst
möglich, wenn die UBs alle Artikel ihrer Zeitschriften einzelnd
verschlagwortet / erfasst hätten. Diesen Service liefern stattdessen
Literaturdatenbanken.

\subsection{Stichwortsuche in Literaturatenbanken}

Literaturdatenbanken sind zumeist fachspezifisch, verschlagworten dafür
aber (auch) die einzelnen Artikel der Zeitschriften. Es gibt Datenbanklisten,
aus denen eine Datenbank auszuwählen ist, um darin zu suchen. Auch die
Datenbanken können frei im Netz zugänglich oder bibliotheksspezifisch sein. In
die Literaturdatenbanklisten kann über verschiedene Links eingestiegen werden:

\subsubsection{Frankfurter Einstiegslinks}
\begin{itemize}
  \item {\ttfamily http://www.ub.uni-frankfurt.de/} $\Rightarrow$ Menueintrag
  \emph{Datenbanken, \ldots}
  \item {\ttfamily http://www.ub.uni-frankfurt.de/banken.html} $\Rightarrow$ 
  \emph{Datenbanken}
  \item Oder direkt {\ttfamily http://info.ub.uni-frankfurt.de/}
\end{itemize}

\subsubsection{Darmstädter Einstiegslinks}
\begin{itemize}
  \item Startseite {\ttfamily http://www.ulb.tu-darmstadt.de/} und Sektion
  \emph{Datenbank-Infosystem (DBIS)} auf der rechte Seitehälfte.
  \item 
  {\ttfamily \small
  \small http://rzblx10.uni-regensburg.de/dbinfo/fachliste.php?bib\_id=tuda}
\end{itemize}

\subsubsection{Wichtige Informatik-Literaturdatenbanken laut UB Frankfurt}

\begin{itemize}
  \item ACM Digital Library / Association for Computing Machinery
  \item CiteSeer.IST Scientific Literature Digital Library
  \item Web of Knowledge \ldots\footnote{ \tiny vgl. {\ttfamily
  http://info.ub.uni-frankfurt.de/fach\_liste.html?fach=informatik }}
\end{itemize}

\subsubsection{Die TOP-10 Informatik-Literaturdatenbanken laut UB Darmstadt}

\begin{itemize}
  \item ACM Digital Library (UB)
  \item arXiv.org e-Print archive (frei)
  \item CiteSeerX Beta 	(frei)
  \item Collection of Computer Science Bibliographies (frei)
  \item Computer Science Bibliography (frei)
  \item WTI-Frankfurt Datenbanken (ehem. FIZ Technik) (UB)
  \item FOLDOC : Free On-Line Dictionary of Computing (frei)
  \item IEEE Xplore / Electronic Library Online (IEL) (UB)
  \item INFODATA (frei)
  \item Journal Citation Reports (UB)
  \item MathSciNet (UB) \ldots
  \item Web of Knowledge - ISI 	(UB)
  \item Web of Science (UB)\ldots\footnote{{\tiny vgl. {\ttfamily 
http://rzblx10.uni-regensburg.de/dbinfo/dbliste.php?bib\_id=tuda\&colors=63\&ocolors=40\&lett=f\&gebiete=30
}}}
\end{itemize}

\section{Fallen}
\begin{itemize}
  \item Es gibt Datenbanken, die über das Internet (CiteSeer, Web of
  Science, ACM Digital Library) oder per Bib-Login (Hebis) {\bfseries in ihrer
  Suchfunktionalität} frei erreichbar sind. Allerdings ist der \textbf{Download des
  Volltextes} bei einigen oder vielen Werken dann doch nur \textbf{über das
  Universitätsnetz} möglich. Deshalb lohnt sich der Einstieg bei einigen von
  ihnen ( ACM Digital Library, Web of Science, Hebis) direkt über das
  Universitätsnetz (Vor-Ort-Recherche).
  \item Erschwert wird der Zugang dadurch, dass für Gast-Rechercheure
  \textbf{bei der Vor-Ort-Recherche} über das UB-Netz der \textbf{Durchgriff
  ins Internet gesperrt} ist.
  Deshalb lohnt es sich bei mehrheitlich frei orientierten Datanbanken
  (CiteSeer) eher, zuerst über das freie Internet zu recherchen und nicht
  zugängliche Downloads gezielt über das Universitätsnetz noch einmal zu suchen
  und dann downzuloaden.
\end{itemize}

\section{Recherche Report}

Auf den folgenden Seiten finden sich drei
Tabellen\footnote{\underline{Relevanz}:
(A) unumgehbar: $\clubsuit$ (B) wichtig: $\spadesuit$ (C) nützlich:
$\heartsuit$ (D) möglich: $\diamondsuit$ - \underline{Ergiebigkeit}: (a) many
items found: $\ast$ (b) some items found: $\star$ (c) little items found:
$\circ$ (d) implicitely searched: ~ (e) nothing found: $\neg$ (f)
ignored/ambigue: $\odot$}, in und mit denen der Recherchestand dokumentiert wird:
\begin{itemize}
  \item die spezifischen Recherche-Resourcen der Unibibliothek Frankfurt, für
  deren Nutzung diese zahlt und die deshalb für Nicht-Universitätsangehörige
  zumeist nur im Netz der Uni-Bibliothek ausgewertet (und die Ergebnisse nur
  ebenso eingesehen) werden können.
  \item die spezifischen Recherche-Resourcen der Universitäts- und
  Landesbibliothek Darmstadt, für deren Nutzung diese zahlt und die deshalb für
  Nicht-Universitätsangehörige zumeist nur im Netz der Uni-Bibliothek
  ausgewertet (und die Ergebnisse nur ebenso eingesehen) werden können.
  \item die frei über das Internet zugänglichen Literaturdatenbanken, die darum
  über das Portal der Universität angesteuert, aber auch außerhalb der
  Uni-Netzes ausgewertet werden können. Die Ergebnisse können wiederum
  elektronische Bücher/Artikel sein, die dann wieder nur im Uni-Netz konkret
  eingesehen werden können.
\end{itemize}

\underline{Erinnerung:} Die Elektronische Zeitschriften Bibliothek (und der
Hebis-Katalog?) bieten nur eine fachbereichsbezogene Suche nach
Zeitschriftentiteln, nicht nach Artikeln in den Zeitschriften. Dafür
sind die einzelnen Datenbanken zuständig\footnote{Name kursiv gesetzt!}!

\underline{Hinweis:} Der Einfachheit halber folgt auf jede Tabelle eine Tabelle
mit den Zeitschriften, die über die EZB/Hebis als relevante Zeitschriftentitel
gefunden worden sind.

\begin{table}
\scriptsize
\caption{Resourcen aus/über Frankfurter Universiätsbibliothek}
\begin{center}
\begin{tabular}[h]{|r|c|c|c||c||c|c|c|c||c|c|c|c|c|c|c|c||c|}
\hline
& \rotatebox{90}{$\clubsuit$ OPAC FaM}
& \rotatebox{90}{$\clubsuit$ \textit{ACM Digital Library}}
& \rotatebox{90}{$\clubsuit$ \textit{Web of Science}}
& \rotatebox{90}{$\spadesuit$ Hebis Portal}
& \rotatebox{90}{$\heartsuit$ \textit{CiteSeer.IST FaM}}
& \rotatebox{90}{$\heartsuit$ \textit{Web of Knowledge}}
& \rotatebox{90}{$\heartsuit$ EZB FaM Informatik}
& \rotatebox{90}{$\heartsuit$ EZB FaM Jura}
& \rotatebox{90}{$\diamondsuit$ \textit{IBZ (Int. Bibl. geistes-\&soz.-wis. Zeitschr.)~}} 
& \rotatebox{90}{$\diamondsuit$ \textit{Int. Philsophiocal Bibl.}} 
& \rotatebox{90}{$\diamondsuit$ \textit{Cambridge Journals Digital Archive}}
& \rotatebox{90}{$\diamondsuit$ \textit{Index to theses (GB/IR)}}
& \rotatebox{90}{$\diamondsuit$ \textit{Springer E-books (Comp. Science + Techn. \& Inf.)~}} 
& \rotatebox{90}{$\diamondsuit$ \textit{Juris Spectrum Datenbank}}
& \rotatebox{90}{$\diamondsuit$ \textit{Oldenbourg \& Akademie e-books (Inf., Phil)}} 
& \rotatebox{90}{$\diamondsuit$ \textit{Oxford Journals}}
& \rotatebox{90}{\itshape{???}}
\\
\hline \hline
Open Source Li[cen[c/s]e]zenz]
  & $\ast$ & $\ast$ & $\ast$ & $\ast$ & ? & $\circ$ 
  & $\circ$ & $\neg$ & ? & ? & ? & ?
  & ? & ? & ? & ? & ?\\
\hline
GNU Public Licen[c/s]e
  & $\star$ & $\star$ & $\circ$ & $\star$ & ? & $\odot$
  & $\neg$ & $\neg$ & ? & ? & ? & ?
  & ? & ? & ? & ? & ?\\
\hline
GNU Lizenz
  & $\circ$ & $\neg$ & $\neg$ & $\circ$ & ? & $\odot$
  & $\neg$ & $\neg$ & ? & ? & ? & ?
  & ? & ? & ? & ? & ?\\
\hline
Free Software Licen[c/s]e]
  & $\circ$ & $\circ$ & $\circ$ & $\circ$ & ? & $\neg$ 
  & $\neg$ & $\neg$ & ? & ? & ? & ?
  & ? & ? & ? & ? & ?\\
\hline
Freie Software Lizenz
  & $\circ$ & $\neg$ & $\neg$ & $\circ$ & ? & $\neg$ 
  & $\neg$ & $\neg$ & ? & ? & ? & ?
  & ? & ? & ? & ? & ?\\
\hline
OSI / Open Source Initiative
  & $\star$ & $\neg$ & $\circ$ & $\neg$ & ? & $\circ$ 
  & $\neg$ & $\neg$ & ? & ? & ? & ?
  & ? & ? & ? & ? & ?\\
\hline
FSF / Free Software Foundation
  & $\star$ & $\star$ & $\neg$ & $\star$ & ? & $\circ$ 
  & $\neg$ & $\neg$ & ? & ? & ? & ?
  & ? & ? & ? & ? & ?\\
\hline
Apache Li[cen[c/s]e]zenz]
  & $\ast$ & $\ast$ & $\neg$ &  $\circ$ & ? & $\odot$
  & $\neg$ & $\neg$ & ? & ? & ? & ?
  & ? & ? & ? & ? & ?\\
\hline
Apache Foundation
  & $\circ$ & $\neg$ & $\neg$ & $\neg$ & ? & $\odot$
  & $\neg$ & $\neg$ & ? & ? & ? & ?
  & ? & ? & ? & ? & ?\\
\hline
Eclipse Public Li[cen[c/s]e]zenz]
  & $\ast$ & $\circ$ & $\circ$ & $\circ$ & ? & $\odot$
  & $\neg$ & $\neg$ & ? & ? & ? & ?
  & ? & ? & ? & ? & ?\\
\hline 
FL/OSS FOSS
  & $\neg$ & $\star$ & $\circ$ & $\neg$ & ? & $\odot$
  & $\neg$ & $\neg$ & ? & ? & ? & ?
  & ? & ? & ? & ? & ?\\
\hline
Copyleft
  & $\circ$ & $\circ$ & $\circ$ & $\star$ & ? & $\odot$
  & $\neg$ & $\neg$ & ? & ? & ? & ?
  & ? & ? & ? & ? & ?\\
\hline
\hline$\odot$
GPL
  & $\circ$ & $\circ$ & $\circ$ & $\neg$ & ? & $\odot$
  & $\neg$ & $\neg$ & ? & ? & ? & ?
  & ? & ? & ? & ? & ?\\
\hline
EPL
  & $\neg$ & $\circ$ & $\neg$ & $\neg$ & ? & $\odot$
  & $\neg$ & $\neg$ & ? & ? & ? & ?
  & ? & ? & ? & ? & ?\\
\hline 
\hline
Open Source
  & $\star$ & $\ast$ & $\odot$ & $\odot$ & ? & $\circ$ 
  & $\circ$ & $\neg$ & ? & ? & ? & ?
  & ? & ? & ? & ? & ?\\
\hline
Free Software
  & $\circ$ & $\circ$ & $\odot$ & $\odot$ & ? & $\circ$
  & $\neg$ & $\neg$ & ? & ? & ? & ?
  & ? & ? & ? & ? & ?\\
\hline
Freie Software
  & $\circ$ & $\neg$ & $\neg$ & $\odot$ & ? & $\neg$
  & $\neg$ & $\neg$ & ? & ? & ? & ?
  & ? & ? & ? & ? & ?\\
\hline
GNU
  & $\circ$ & $\circ$ & $\odot$ & $\odot$ & ? & $\odot$
  & $\neg$ & $\neg$ & ? & ? & ? & ?
  & ? & ? & ? & ? & ?\\
\hline
Apache
  & $\circ$ & $\neg$ & $\neg$ & $\odot$ & ? & $\odot$
  & $\neg$ & $\neg$ & ? & ? & ? & ?
  & ? & ? & ? & ? & ?\\
\hline
Eclipse
  & $\neg$ & $\circ$ & $\odot$ & $\odot$ & ? & $\odot$
  & $\circ$ & $\neg$ & ? & ? & ? & ?
  & ? & ? & ? & ? & ?\\
\hline
\hline
OSI
  & $\ast$ & $\neg$ & - & - & ? & -
  & $\neg$ & $\neg$ & ? & ? & ? & ?
  & ? & ? & ? & ? & ?\\
\hline
FLOSS
  & - & - & - & - & ? & -
  & - & - & ? & ? & ? & ?
  & ? & ? & ? & ? & ?\\
\hline
OSS
  & $\ast$ & $\ast$ & - & - & ? & -
  & $\neg$ & $\neg$ & ? & ? & ? & ?
  & ? & ? & ? & ? & ?\\
\hline

\end{tabular}
\end{center}
\end{table}



\begin{table}
\scriptsize
\caption{dazu EZB/HEBIS gefundene Zeitschriftentitel \& (Konferenz-)Reihen}
\begin{center}
\begin{tabular}[h]{|r|l|l|l|}
\hline 
1 & ZS & Communications of the ACM & \\
\hline 
2 & ZS & Open Source Journal & http://www.opensourcejournal.ro/\\
\hline 
3 & CR & FLOSS 08 - 11 & \\
\hline 
4 & ZS & & \\
\hline 
5 & ZS & & \\
\hline 
\hline
\end{tabular}
\end{center}
\end{table}



\begin{table}
\small
\caption{Spezifische Resourcen der Darmstdätter Uni- u.Landesbibliothek}
\begin{center}
\begin{tabular}[h]{|r|c|c|c||c|c|c||c|c||c|c|c|}
\hline
& \rotatebox{90}{$\clubsuit$ OPAC DA}
& \rotatebox{90}{$\clubsuit$ DAKAPO DA}
& \rotatebox{90}{$\clubsuit$ \textit{IEEE Xplore / Electr. Libr. Online DA}}
& \rotatebox{90}{$\spadesuit$ Hebis Portal}
& \rotatebox{90}{$\spadesuit$ \textit{ACM Digital Library DA/FaM}}
& \rotatebox{90}{$\spadesuit$ \textit{Web of Science (DA)}}
& \rotatebox{90}{$\heartsuit$ \textit{Web of Knowledge (DA/FaM)}}
& \rotatebox{90}{$\heartsuit$ EZB DA}
& \rotatebox{90}{$\diamondsuit$ \textit{Journal Citation Reports (DA)}}
& \rotatebox{90}{$\diamondsuit$ \textit{MathSciNet (DA)}}
& \rotatebox{90}{$\diamondsuit$ \textit{WTI-FaM Datenbanken FIZ Technik DA}}
\\
\hline \hline
Open Source Li[cen[c/s]e]zenz]
  & ? & ? & ? & ? & ? & ? 
  & ? & ? & ? & ? & ?\\
\hline
GNU Public Licen[c/s]e
  & ? & ? & ? & ? & ? & ? 
  & ? & ? & ? & ? & ?\\
\hline
GNU Lizenz
  & ? & ? & ? & ? & ? & ? 
  & ? & ? & ? & ? & ?\\
\hline
Free Software Licen[c/s]e]
  & ? & ? & ? & ? & ? & ? 
  & ? & ? & ? & ? & ?\\
\hline
Freie Software Lizenz & ?
  & ? & ? & ? & ? & ? 
  & ? & ? & ? & ? & ?\\
\hline
OSI / Open Source Initiative
  & ? & ? & ? & ? & ? & ? 
  & ? & ? & ? & ? & ?\\
\hline
FSF / Free Software Foundation
  & ? & ? & ? & ? & ? & ? 
  & ? & ? & ? & ? & ?\\
\hline
Apache Li[cen[c/s]e]zenz]
  & ? & ? & ? & ? & ? & ? 
  & ? & ? & ? & ? & ?\\
\hline
Apache Foundation
  & ? & ? & ? & ? & ? & ? 
  & ? & ? & ? & ? & ?\\
\hline
Eclipse Public Li[cen[c/s]e]zenz]
  & ? & ? & ? & ? & ? & ? 
  & ? & ? & ? & ? & ?\\
\hline 
FL/OSS FOSS
  & ? & ? & ? & ? & ? & ? 
  & ? & ? & ? & ? & ?\\
\hline
Copyleft
  & ? & ? & ? & ? & ? & ? 
  & ? & ? & ? & ? & ?\\
\hline
\hline
GPL
  & ? & ? & ? & ? & ? & ? 
  & ? & ? & ? & ? & ?\\
\hline
EPL
  & ? & ? & ? & ? & ? & ? 
  & ? & ? & ? & ? & ?\\
\hline 
\hline
Open Source
  & ? & ? & ? & ? & ? & ? 
  & ? & ? & ? & ? & ?\\
\hline
Free Software
  & ? & ? & ? & ? & ? & ? 
  & ? & ? & ? & ? & ?\\
\hline
Freie Software
  & ? & ? & ? & ? & ? & ? 
  & ? & ? & ? & ? & ?\\
\hline
GNU
  & ? & ? & ? & ? & ? & ? 
  & ? & ? & ? & ? & ?\\
\hline
Apache
  & ? & ? & ? & ? & ? & ? 
  & ? & ? & ? & ? & ?\\
\hline
Eclipse
  & ? & ? & ? & ? & ? & ? 
  & ? & ? & ? & ? & ?\\
\hline
\hline
OSI
  & ? & ? & ? & ? & ? & ? 
  & ? & ? & ? & ? & ?\\
\hline
FLOSS
  & ? & ? & ? & ? & ? & ? 
  & ? & ? & ? & ? & ?\\
\hline
OSS
  & ? & ? & ? & ? & ? & ?
  & ? & ? & ? & ? & ?\\
\hline
\end{tabular}
\end{center}
\end{table}


\begin{table}
\small
\caption{Freie Internetresourcen (ggfls. redo UB-Portale DA oder FaM)}
\begin{center}
\begin{tabular}[h]{|r||c|c|c||c|c||c|c|c|c|c|c||c|c|c|}
\hline
& \rotatebox{90}{$\clubsuit$ \textit{CiteSeer}}
& \rotatebox{90}{$\clubsuit$ Amazon Book Store}
& \rotatebox{90}{$\clubsuit$ O'Reilly Book Store}
& \rotatebox{90}{$\spadesuit$ \textit{bibsonomy}}
& \rotatebox{90}{$\spadesuit$ \textit{Google Scholar}}
& \rotatebox{90}{$\heartsuit$ \textit{AarXiv.org [DA]}}
& \rotatebox{90}{$\heartsuit$ \textit{Coll. of Comp. Science Bibl. [DA]}}
& \rotatebox{90}{$\heartsuit$ \textit{Comp. Science Bibliography [DA]}}
& \rotatebox{90}{$\heartsuit$ \textit{(F)ree (O)n-(L)ine (D)ictionary (o)f (C)omputing [DA]}}
& \rotatebox{90}{$\heartsuit$ \textit{INFODATA [DA]}}
& \rotatebox{90}{$\heartsuit$ \textit{spires}}
& \rotatebox{90}{$\diamondsuit$ Hebis}
& \rotatebox{90}{$\diamondsuit$ \textit{Web of Science}}
& \rotatebox{90}{$\diamondsuit$ \textit{Web of Knowledge}}
\\
\hline \hline
Open Source Li[cen[c/s]e]zenz]
  & ? & ? & ? & ? & ? 
  & ? & ? & ? & ? & ? 
  & ? & ? & ? & ?\\
\hline
GNU Public Licen[c/s]e
  & ? & ? & ? & ? & ? 
  & ? & ? & ? & ? & ? 
  & ? & ? & ? & ?\\
\hline
GNU Lizenz
  & ? & ? & ? & ? & ? 
  & ? & ? & ? & ? & ? 
  & ? & ? & ? & ?\\
\hline
Free Software Licen[c/s]e]
  & ? & ? & ? & ? & ? 
  & ? & ? & ? & ? & ? 
  & ? & ? & ? & ?\\
\hline
Freie Software Lizenz
  & ? & ? & ? & ? & ? 
  & ? & ? & ? & ? & ? 
  & ? & ? & ? & ?\\
\hline
OSI / Open Source Initiative
  & ? & ? & ? & ? & ? 
  & ? & ? & ? & ? & ? 
  & ? & ? & ? & ?\\
\hline
FSF / Free Software Foundation
  & ? & ? & ? & ? & ? 
  & ? & ? & ? & ? & ? 
  & ? & ? & ? & ?\\
hline
Apache Li[cen[c/s]e]zenz]
  & ? & ? & ? & ? & ? 
  & ? & ? & ? & ? & ? 
  & ? & ? & ? & ?\\
\hline
Apache Foundation
  & ? & ? & ? & ? & ? 
  & ? & ? & ? & ? & ? 
  & ? & ? & ? & ?\\
\hline
Eclipse Public Li[cen[c/s]e]zenz]
  & ? & ? & ? & ? & ? 
  & ? & ? & ? & ? & ? 
  & ? & ? & ? & ?\\
\hline 
FL/OSS FOSS
  & ? & ? & ? & ? & ? 
  & ? & ? & ? & ? & ? 
  & ? & ? & ? & ?\\
\hline
Copyleft
  & ? & ? & ? & ? & ? 
  & ? & ? & ? & ? & ? 
  & ? & ? & ? & ?\\
\hline
\hline
GPL
  & ? & ? & ? & ? & ? 
  & ? & ? & ? & ? & ? 
  & ? & ? & ? & ?\\
\hline
EPL
  & ? & ? & ? & ? & ? 
  & ? & ? & ? & ? & ? 
  & ? & ? & ? & ?\\
\hline 
\hline
Open Source
  & ? & ? & ? & ? & ? 
  & ? & ? & ? & ? & ? 
  & ? & ? & ? & ?\\
\hline
Free Software
  & ? & ? & ? & ? & ? 
  & ? & ? & ? & ? & ? 
  & ? & ? & ? & ?\\
\hline
Freie Software
  & ? & ? & ? & ? & ? 
  & ? & ? & ? & ? & ? 
  & ? & ? & ? & ?\\
\hline
GNU
  & ? & ? & ? & ? & ? 
  & ? & ? & ? & ? & ? 
  & ? & ? & ? & ?\\
\hline
Apache
  & ? & ? & ? & ? & ? 
  & ? & ? & ? & ? & ? 
  & ? & ? & ? & ?\\
\hline
Eclipse
  & ? & ? & ? & ? & ? 
  & ? & ? & ? & ? & ? 
  & ? & ? & ? & ?\\
\hline
\hline
OSI
  & ? & ? & ? & ? & ? 
  & ? & ? & ? & ? & ? 
  & ? & ? & ? & ?\\
\hline
FLOSS
  & ? & ? & ? & ? & ? 
  & ? & ? & ? & ? & ? 
  & ? & ? & ? & ?\\
\hline
OSS
  & ? & ? & ? & ? & ? 
  & ? & ? & ? & ? & ? 
  & ? & ? & ? & ?\\
\hline
\end{tabular}
\end{center}
\end{table}
\newpage
\small
\bibliography{../bibfiles/oscXtras}

\end{document}
