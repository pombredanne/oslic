% Telekom osCompendium extract template
%
% (c) Karsten Reincke, Deutsche Telekom AG, Darmstadt 2011
%
% This LaTeX-File is licensed under the Creative Commons Attribution-ShareAlike
% 3.0 Germany License (http://creativecommons.org/licenses/by-sa/3.0/de/): Feel
% free 'to share (to copy, distribute and transmit)' or 'to remix (to adapt)'
% it, if you '... distribute the resulting work under the same or similar
% license to this one' and if you respect how 'you must attribute the work in
% the manner specified by the author ...':
%
% In an internet based reuse please link the reused parts to www.telekom.com and
% mention the original authors and Deutsche Telekom AG in a suitable manner. In
% a paper-like reuse please insert a short hint to www.telekom.com and to the
% original authors and Deutsche Telekom AG into your preface. For normal
% quotations please use the scientific standard to cite.
%
% [ File structure derived from 'mind your Scholar Research Framework' 
%   mycsrf (c) K. Reincke CC BY 3.0  http://mycsrf.fodina.de/ ]

% select the document class
% S.26: [ 10pt|11pt|12pt onecolumn|twocolumn oneside|twoside notitlepage|titlepage final|draft
%         leqno fleqn openbib a4paper|a5paper|b5paper|letterpaper|legalpaper|executivepaper openrigth ]
% S.25: { article|report|book|letter ... }
%
% oder koma-skript S.10 + 16
\documentclass[DIV=calc,BCOR=5mm,11pt,headings=small,oneside,abstract=true, toc=bib]{scrartcl}

%%% (1) general configurations %%%
\usepackage[utf8]{inputenc}

%%% (2) language specific configurations %%%
\usepackage[]{a4,ngerman}
\usepackage[english, ngerman]{babel}
\selectlanguage{ngerman}

%language specific quoting signs
%default for language emglish is american style of quotes
\usepackage{csquotes}

% jurabib configuration
\usepackage[see]{jurabib}
\bibliographystyle{jurabib}
% Telekom osCompendium German Jurabib Configuration Include Module 
%
% (c) Karsten Reincke, Deutsche Telekom AG, Darmstadt 2011
%
% This LaTeX-File is licensed under the Creative Commons Attribution-ShareAlike
% 3.0 Germany License (http://creativecommons.org/licenses/by-sa/3.0/de/): Feel
% free 'to share (to copy, distribute and transmit)' or 'to remix (to adapt)'
% it, if you '... distribute the resulting work under the same or similar
% license to this one' and if you respect how 'you must attribute the work in
% the manner specified by the author ...':
%
% In an internet based reuse please link the reused parts to www.telekom.com and
% mention the original authors and Deutsche Telekom AG in a suitable manner. In
% a paper-like reuse please insert a short hint to www.telekom.com and to the
% original authors and Deutsche Telekom AG into your preface. For normal
% quotations please use the scientific standard to cite.
%
% [ File structure derived from 'mind your Scholar Research Framework' 
%   mycsrf (c) K. Reincke CC BY 3.0  http://mycsrf.fodina.de/ ]

% the first time cite with all data, later with shorttitle
\jurabibsetup{citefull=first}

%%% (1) author / editor list configuration
%\jurabibsetup{authorformat=and} % uses 'und' instead of 'u.'
% therefore define your own abbreviated conjunction: 
% an 'and before last author explicetly written conjunction

% for authors in citations
\renewcommand*{\jbbtasep}{ u. } % bta = between two authors sep
\renewcommand*{\jbbfsasep}{, } % bfsa = between first and second author sep
\renewcommand*{\jbbstasep}{ u. }% bsta = between second and third author sep
% for editors in citations
\renewcommand*{\jbbtesep}{ u. } % bta = between two authors sep
\renewcommand*{\jbbfsesep}{, } % bfsa = between first and second author sep
\renewcommand*{\jbbstesep}{ u. }% bsta = between second and third author sep

% for authors in literature list
\renewcommand*{\bibbtasep}{ u. } % bta = between two authors sep
\renewcommand*{\bibbfsasep}{, } % bfsa = between first and second author sep
\renewcommand*{\bibbstasep}{ u. }% bsta = between second and third author sep
% for editors  in literature list
\renewcommand*{\bibbtesep}{ u. } % bte = between two editors sep
\renewcommand*{\bibbfsesep}{, } % bfse = between first and second editor sep
\renewcommand*{\bibbstesep}{ u. }% bste = between second and third editor sep

% use: name, forname, forname lastname u. forname lastname
\jurabibsetup{authorformat=firstnotreversed}
\jurabibsetup{authorformat=italic}

%%% (2) title configuration
% in every case print the title, let it be seperated from the 
% author by a colon and use the slanted font
\jurabibsetup{titleformat={all,colonsep}}
%\renewcommand*{\jbtitlefont}{\textit}

%%% (3) seperators in bib data
% separate bibliographical hints and page hints by a comma
\jurabibsetup{commabeforerest}

%%% (4) specific configuration of bibdata in quotes / footnote
% use a.a.O if possible
\jurabibsetup{ibidem=strict}

% replace ugly a.a.O. by ders., a.a.O. resp. ders., ebda.
% but if there are more than one author or girl writers?
\AddTo\bibsgerman{
  \renewcommand*{\ibidemname}{Ds., a.a.O.}
  \renewcommand*{\ibidemmidname}{ds., a.a.O.}
}
\renewcommand*{\samepageibidemname}{Ds., ebda.}
\renewcommand*{\samepageibidemmidname}{ds., ebda.}

%%% (5) specific configuration of bibdata in bibliography
% ever an in: before journal and collection/book-tiltes 
\renewcommand*{\bibbtsep}{in: }
%\renewcommand*{\bibjtsep}{in: }

% ever a colon after author names 
\renewcommand*{\bibansep}{: }
% ever a semi colon after the title 
\renewcommand*{\bibatsep}{; }
% ever a comma before date/year
\renewcommand*{\bibbdsep}{, }

% let jurabib insert the S. and p. information
% no S. necessary in bib-files and in cites/footcites
\jurabibsetup{pages=format}

% use a compressed literature-list using a small line indent
\jurabibsetup{bibformat=compress}
\setlength{\jbbibhang}{1em}

% which follows the design of the cites and offers comments
\jurabibsetup{biblikecite}

% print annotations into bibliography
\jurabibsetup{annote}
\renewcommand*{\jbannoteformat}[1]{{ \itshape #1 }}

%refine the prefix of url download
\AddTo\bibsgerman{\renewcommand*{\urldatecomment}{Referenzdownload: }}

% we want to have the year of articles in brackets
\renewcommand*{\bibaldelim}{(}
\renewcommand*{\bibardelim}{)}

%Umformatierung des Reihentitels und der Reihennummer
\DeclareRobustCommand{\numberandseries}[2]{%
\unskip\unskip%,
\space\bibsnfont{(=~#2}%
\ifthenelse{\equal{#1}{}}{)}{, [Bd./Nr.]~#1)}%
}%


% language specific hyphenation
% Telekom osCompendium osHyphenation Include Module
%
% (c) Karsten Reincke, Deutsche Telekom AG, Darmstadt 2011
%
% This LaTeX-File is licensed under the Creative Commons Attribution-ShareAlike
% 3.0 Germany License (http://creativecommons.org/licenses/by-sa/3.0/de/): Feel
% free 'to share (to copy, distribute and transmit)' or 'to remix (to adapt)'
% it, if you '... distribute the resulting work under the same or similar
% license to this one' and if you respect how 'you must attribute the work in
% the manner specified by the author ...':
%
% In an internet based reuse please link the reused parts to www.telekom.com and
% mention the original authors and Deutsche Telekom AG in a suitable manner. In
% a paper-like reuse please insert a short hint to www.telekom.com and to the
% original authors and Deutsche Telekom AG into your preface. For normal
% quotations please use the scientific standard to cite.
%
% [ File structure derived from 'mind your Scholar Research Framework' 
%   mycsrf (c) K. Reincke CC BY 3.0  http://mycsrf.fodina.de/ ]
%


\hyphenation{rein-cke}




%%% (3) layout page configuration %%%

% select the visible parts of a page
% S.31: { plain|empty|headings|myheadings }
%\pagestyle{myheadings}
\pagestyle{headings}

% select the wished style of page-numbering
% S.32: { arabic,roman,Roman,alph,Alph }
\pagenumbering{arabic}
\setcounter{page}{1}

% select the wished distances using the general setlength order:
% S.34 { baselineskip| parskip | parindent }
% - general no indent for paragraphs
\setlength{\parindent}{0pt}
\setlength{\parskip}{1.2ex plus 0.2ex minus 0.2ex}


%%% (4) general package activation %%%
%\usepackage{utopia}
%\usepackage{courier}
%\usepackage{avant}
\usepackage[dvips]{epsfig}

% graphic
\usepackage{graphicx,color}
\usepackage{array}
\usepackage{shadow}
\usepackage{fancybox}

%- start(footnote-configuration)
%  flush the cite numbers out of the vertical line and let
%  the footnote text directly start in the left vertical line
\usepackage[marginal]{footmisc}
%- end(footnote-configuration)

\begin{document}

%% use all entries of the bliography

%%-- start(titlepage)
\titlehead{Literaturexzerpt}
\subject{Autor(en): Mundhenke}
\title{Titel: Wettbewerbswirkungen von Open-Source-Software und offenen
Standards auf Softwaremärkten}
\subtitle{Jahr: 2007 }
\author{K. Reincke% Telekom osCompendium License Include Module
%
% (c) Karsten Reincke, Deutsche Telekom AG, Darmstadt 2011
%
% This LaTeX-File is licensed under the Creative Commons Attribution-ShareAlike
% 3.0 Germany License (http://creativecommons.org/licenses/by-sa/3.0/de/): Feel
% free 'to share (to copy, distribute and transmit)' or 'to remix (to adapt)'
% it, if you '... distribute the resulting work under the same or similar
% license to this one' and if you respect how 'you must attribute the work in
% the manner specified by the author ...':
%
% In an internet based reuse please link the reused parts to www.telekom.com and
% mention the original authors and Deutsche Telekom AG in a suitable manner. In
% a paper-like reuse please insert a short hint to www.telekom.com and to the
% original authors and Deutsche Telekom AG into your preface. For normal
% quotations please use the scientific standard to cite.
%
% [ File structure derived from 'mind your Scholar Research Framework' 
%   mycsrf (c) K. Reincke CC BY 3.0  http://mycsrf.fodina.de/ ]
%
\footnote{
This text is licensed under the Creative Commons Attribution-ShareAlike 3.0 Germany
License (http://creativecommons.org/licenses/by-sa/3.0/de/): Feel free \enquote{to
share (to copy, distribute and transmit)} or \enquote{to remix (to
adapt)} it, if you \enquote{[\ldots] distribute the resulting work under the
same or similar license to this one} and if you respect how \enquote{you
must attribute the work in the manner specified by the author(s)
[\ldots]}):
\newline
In an internet based reuse please mention the initial authors in a suitable
manner, name their sponsor \textit{Deutsche Telekom AG} and link it to
\texttt{http://www.telekom.com}. In a paper-like reuse please insert a short
hint to \texttt{http://www.telekom.com}, to the initial authors, and to their
sponsor \textit{Deutsche Telekom AG} into your preface. For normal citations
please use the scientific standard.
\newline
{ \tiny \itshape [derived from myCsrf (= 'mind your Scholar Research Framework') 
\copyright K. Reincke CC BY 3.0  http://mycsrf.fodina.de/)] }}}

%\thanks{den Autoren von KOMA-Script und denen von Jurabib}
\maketitle
%%-- end(titlepage)
%\nocite{*}

\begin{abstract}
\noindent
Das Werk / The work\footcite[][]{Mundhenke2007a} \\
\noindent \itshape
\ldots Fragt, warum und wie Open Source Software erfolgreich im Markt agiert.
Eine Antwort sei, dass OSS den Wettbewerbsdruck durch die blosse Existenz einer
Alternative erhöht. Zur Analyse der Frage folgt das Buch der 'natürlichen'
thematischen Struktur: es erläutert das Konzept Open Source, klassiziert die
Lizenzen in Copyleft-, Noncopyleft-, Public Domain- und proprietäre Lizenzen und
skizziert die Geschichte von Open Source.\\
\noindent
\ldots This book asks why and how Open Source Software successfully works as
part of the economic markets. One answer shall be that OSS increments the
competitive pressure by offering an alternative. Fort analyzing the topic the
book follows the 'natural' structure of handling Open Source: it explains the
concept, classifies the licenses as Copyleft, Noncopyleft, Public Domain and
proprietary, and gives a summary of the OSS history.
\end{abstract}
\footnotesize
%\tableofcontents
\normalsize

\section{Line of Thought}

\subsection{Leitfragen}

Ausgangspunkt der Arbeit ist die Überlegung, dass Open Source Software -
spätestens mit der Freigabe des Browser-Quellcodes durch Netscape - immer
eingebunden war in einen
Verdrängungswettkampf\footcite[vgl.][1]{Mundhenke2007a}, wobei das Gebahren
eines der Teilnehmer, Microsoft, schon seit längerem von den
\enquote{US-Wettbewerbsbehörden} untersucht und schließlich \enquote{im Juni
2000} als \enquote{umfassender Missbrauch von Marktmacht}
verurteilt worden ist\footcite[vgl.][2]{Mundhenke2007a}

Auf dieser Basis geht das Buch mit 5 Leitfragen ins Rennen: Es will die Fragen
beantworten, warum (a) \enquote{[\ldots] das Betriebssystem Linux von einer
weltweiten Community entwickelt und im Internet lizenzkostenfrei [\ldots]
angeboten (werde)}, warum (b) diese Software \enquote{[\ldots] nahezu
uneingeschränkt genutzt und weitergegeben werden (dürfe)}, wie (c)
\enquote{[\ldots] trotz eines solch unkonventionellen Entwicklungsmodells
qualitativ herausragende Software entstehen (könne)}, warum mittlerweile
(d) \enquote{[\ldots] fast alle namhaften IT-Unternehmen nicht nur als
Nutzer, sondern als Förderer und Anbieter erhebliche Resourcen in die
EDntwicklung von Open-Source-Software (investieren)} und warum sie
(e) dieses auch tun, \enquote{ohne [dabei] gesicherte Eigentumsansprüche zu
erwerben}\footcite[vgl.][3]{Mundhenke2007a}

Zusätzlich möchte das Buch der Frage nachgehen, \enquote{[\ldots] ob die
positiven Wirkungen von Open-Source-Software durch gezielte
wirtschaftspolitische Maßnahmen verstärkt werden können und
sollten}\footcite[vgl.][3]{Mundhenke2007a}

\subsection{Argumentation}

\begin{itemize}
  \item Zunächst wird der Softwaremarkt als solcher
  analysiert\footcite[vgl.][21ff]{Mundhenke2007a}, und zwar mit dem Ergebnis,
  dass Softwaremärkte - trotz ihres Hanges zum \enquote{natürlichen Monopol}
  - \enquote{[\ldots] ein Beispiel für Märkte (darstellen), die prinzipiell
  das Merkmal der Bestreitbarkeit aufweisen, auch wenn durch (temporäre)
  Marktbarrieren die Offenheit des Wettbewerbs eingeschränkt (seien)}.
  Durch die \enquote{hohe Dynamik des technischen Wandels} seien
  \enquote{[\ldots] Marktanteile nur bedingt aussagekräftig zur Beurteilung
  der Wettbewerbsintensität [\ldots]}\footcite[vgl.][3]{Mundhenke2007a}
  \item Sodann skizziert das Buch das
  \enquote{Open-Source-Monzept}\footcite[vgl.][40ff -
  Einzelheiten und Spezifisches]{Mundhenke2007a},  klassifiziert
  Software-Lizenzen\footcite[vgl.][48f -
  Einzelheiten und Spezifisches]{Mundhenke2007a} und skizziert die Geschichte
  von OSS\footcite[vgl.][50ff -
  Einzelheiten und Spezifisches]{Mundhenke2007a}. Beispielhaft aufgelistete Open
  Source Projekte sollen dann die \enquote{marktliche Relevanz}
  belegen\footcite[vgl.][54ff]{Mundhenke2007a}. Daran schließen sich noch
  Überlegungen zur statistischen Erfassung und Vergleichbarkeit
  an\footcite[vgl.][65ff]{Mundhenke2007a}
  \item Sodann wird analysiert, wie sich Open Source Projekte aus Sicht der
  Ökonomie darstellen\footcite[vgl.][69ff]{Mundhenke2007a} und \enquote{[\ldots]
  insebsondere das vermeintliche Anreiz- und das
  Koordinationsproblem der Open-Source-Entwicklung
  diskutiert}\footcite[vgl.][117f]{Mundhenke2007a}. Auch für diese Art
  der Arbeit sei zu konstieren, \enquote{[\ldots] dass die
  individuelle Beiteiligung an Open-Source-Projekten überwiegend unter
  dem Aspekt des persönlichen Nutzens subsumiert werden (könne)} und dass
  \enquote{die Organisation der Open-Source-Projekte [\ldots] eine
  deutliche Arbeitsteiliung (zeige) und [\ldots] ausgeprägte
  hierarchische Strukturen (zeige), welche das Bild eines unkoordinierten
  orientalischen Basars widerlegen}\footcite[vgl.][118]{Mundhenke2007a}.
  Im Hinblick auf die Frage nach der Förderungswürdigkeit meinen die Autoren,
  dass \enquote{[\ldots] sich staatliches Handeln darauf beschränken
  (solle), neutrale Rahmenbedingungen vorzugeben, die einen offenen
  Wettbewerb von Open-Source-Software und proprietärer Software im Sinne
  eines bestreitbaren Marktes
  gewährleisten}\footcite[vgl.][119]{Mundhenke2007a}
  \item Danach geht das Buch den \enquote{Markthemmnissen und
  Einflussfaktoren für einen offenen Wettbewerb}
  nach\footcite[vgl.][120ff]{Mundhenke2007a}. Es kommt zu dem Schluss, dass Open
  Source Software \enquote{arbeitsteilig} erstellt werden, und zwar
  ebenso durch \enquote{freie (Freizeit-) Entwickler}, wie durch
  \enquote{gewinnorientierte
  Unternehmen}\footcite[vgl.][165]{Mundhenke2007a}. Die unternehmerische
  Entscheidung für die Verwendung von Open Source Software sei ihrer
  \enquote{qualitativen Hochwertigkeit} geschuldet. Als
  \enquote{bedeutsames Nachfragehemmnis} wirke dagegen, dass \enquote{[\ldots]
  erhebliche Informationsdefizite und Unsicherheiten über die
  Funktionsweise und die Leistungsfähigkeit des Open-Source-Modells
  (bestehen) [\ldots]}\footcite[vgl.][165]{Mundhenke2007a}. Einen
  \enquote{aktiven Handlungsbedarf}\footcite[vgl.][165]{Mundhenke2007a}
  sehen die Autoren dagen bei dem Staat als Auftraggeber, der ob seiner
  Marktmacht immer auch aufpassen müsse, das Gefüge von proprietärer und freie
  Software nicht zu (zer)stören\footcite[vgl.][165f]{Mundhenke2007a}: Dienlich
  sei dazu aber nicht eine \enquote{generelle Privilegierung von
  Open-Source-Software}, sondern die Verwendung von \enquote{offenen
  Standards}\footcite[vgl.][166]{Mundhenke2007a}
  \item Schließlich wird analysiert, wodurch sich offenen Standards
  auszeichnen\footcite[vgl.][168ff]{Mundhenke2007a}: Im zentralen Sinne folgen
  sie der Idee von Open Source Software, wenn offene Standards definiert werden
  über deren  \enquote{einfache Zugänglichkeit}, deren
  \enquote{diskriminierungsfreie} und \enquote{lizenzkostenfreie
  Nutzung} und deren \enquote{lizenzkostenfreie
  Weitergabe}\footcite[vgl.][175]{Mundhenke2007a}. Gleichwohl müsse
  konstatiert werden, dass sich bisher \enquote{wegen der
  Mehrdimensionalität des Offenheitskriteriums [\ldots] noch kein
  einheitliches Konzept für offene Standards herausgebildet
  (habe)}\footcite[vgl.][189]{Mundhenke2007a}
\end{itemize}

\subsection{Gesamtfazit}
\begin{itemize}
  \item Fakt sei, dass \enquote{Open-Source-Software [\ldots] den
  Kidenschuhen entwaschen (sei) und [\ldots] sich im Markt etabliert
  (haben)}\footcite[vgl.][223]{Mundhenke2007a}.
  \item Die \enquote{Motivation}, Open Source Software zu erstellen,
  \enquote{[\ldots] (spiegele) überwiegend ein ökonomisch rationales
  Verhalten (wider) und (sei) nur zu einem kleinen Teil auf übergeordnete
  gesellschaftsphilosophische und soziale Zielsetzungen zurückzuführen
  [\ldots]}, und zwar bei Entwicklern und
  Unternehmen\footcite[vgl.][225]{Mundhenke2007a}
  \item Übergreifend gelte, dass \enquote{das Aufkommen von
  Open-Source-Software [\ldots] mit einer direkten Intensivierung des
  Wettbewerbs (einhergehe)} und dass \enquote{[\ldots]
  Open-Source-Sofwtare [\ldots] (dazu beitrage), die Bestreitbarkeit von
  Softwaremärkten zu erhöhen}\footcite[vgl.][225]{Mundhenke2007a}
  \item Wiederholt wird hier nochmal: Die unternehmerische
  Entscheidung für die Verwendung von Open Source Software sei ihrer
  \enquote{qualitativen Hochwertigkeit} geschuldet. Als
  \enquote{bedeutsames Nachfragehemmnis} wirke dagegen, dass \enquote{[\ldots]
  erhebliche Informationsdefizite und Unsicherheiten über die
  Funktionsweise und die Leistungsfähigkeit des Open-Source-Modells
  (bestehen) [\ldots]}\footcite[vgl.][227]{Mundhenke2007a}
\end{itemize}


\section{Specific Aspects}

Mundhenke folgt dem natürlich Topos der Open Source Behandlung. Zuerst definiert
er Begriffe und grenzt sie gegen benachbarte ab, dann klassifiziert er die
Lizenzmodelle und schließlich skizziert er die Genese der Idee von Open Source:

\subsection{Begriffsdefintionen}

\subsection{Lizenzklassifikation}

\subsection{Zur Genese des \enquote{Open Soiurce Modells}}

Mundhenke liefert bei seiner Beschreibung der Genese eine sehr konzentierte und
gut Darstellung der Hauptlinien:

\begin{itemize}
  \item \enquote{Software als Beigut zu Computern}: Am Anfang in den 50er
  und 60er Jahren sei 'natürlich' gewesen, dass die Software an bestimmte
  Rechner gekoppelt gewesen und mit diesen ganzheitlich ausgeliefert worden sei,
  und zwar auch als Sourcecode, und nicht nur in binärer Form. Auf dieser Basis
  entstand in diesen Jahren die Kultur des freien Austauschs von Verbesserungen
  und Weiterentwicklungen des \enquote{Beiguts Software}. Aufgehoben worden
  sei diese \enquote{Bündelung von Hard- und Software} erst
  \enquote{1969}, um durch eine solche \enquote{Entflechtung} ein
  \enquote{Kartellverfahren} gegen IBM
  entgegenzuwirken\footcite[vgl.][50]{Mundhenke2007a}
  \item \enquote{Unix und das Revival des offenen Codes}: Die zweite Säule
  der Tradition der freien Software ist an die Kartellrechtsgeschichte von Unix
  gebunden.
\end{itemize}

\small
\bibliography{../bibfiles/oscResourcesDe}

\end{document}
