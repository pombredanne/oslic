% Telekom osCompendium extract template
%
% (c) Karsten Reincke, Deutsche Telekom AG, Darmstadt 2011
%
% This LaTeX-File is licensed under the Creative Commons Attribution-ShareAlike
% 3.0 Germany License (http://creativecommons.org/licenses/by-sa/3.0/de/): Feel
% free 'to share (to copy, distribute and transmit)' or 'to remix (to adapt)'
% it, if you '... distribute the resulting work under the same or similar
% license to this one' and if you respect how 'you must attribute the work in
% the manner specified by the author ...':
%
% In an internet based reuse please link the reused parts to www.telekom.com and
% mention the original authors and Deutsche Telekom AG in a suitable manner. In
% a paper-like reuse please insert a short hint to www.telekom.com and to the
% original authors and Deutsche Telekom AG into your preface. For normal
% quotations please use the scientific standard to cite.
%
% [ File structure derived from 'mind your Scholar Research Framework' 
%   mycsrf (c) K. Reincke CC BY 3.0  http://mycsrf.fodina.de/ ]

%
% select the document class
% S.26: [ 10pt|11pt|12pt onecolumn|twocolumn oneside|twoside notitlepage|titlepage final|draft
%         leqno fleqn openbib a4paper|a5paper|b5paper|letterpaper|legalpaper|executivepaper openrigth ]
% S.25: { article|report|book|letter ... }
%
% oder koma-skript S.10 + 16
\documentclass[DIV=calc,BCOR=5mm,11pt,headings=small,oneside,abstract=true, toc=bib]{scrartcl}

%%% (1) general configurations %%%
\usepackage[utf8]{inputenc}

%%% (2) language specific configurations %%%
\usepackage[]{a4,ngerman}
\usepackage[english, ngerman]{babel}
\selectlanguage{ngerman}

%language specific quoting signs
%default for language emglish is american style of quotes
\usepackage[german=quotes]{csquotes}

% jurabib configuration
\usepackage[see]{jurabib}
\bibliographystyle{jurabib}
% Telekom osCompendium German Jurabib Configuration Include Module 
%
% (c) Karsten Reincke, Deutsche Telekom AG, Darmstadt 2011
%
% This LaTeX-File is licensed under the Creative Commons Attribution-ShareAlike
% 3.0 Germany License (http://creativecommons.org/licenses/by-sa/3.0/de/): Feel
% free 'to share (to copy, distribute and transmit)' or 'to remix (to adapt)'
% it, if you '... distribute the resulting work under the same or similar
% license to this one' and if you respect how 'you must attribute the work in
% the manner specified by the author ...':
%
% In an internet based reuse please link the reused parts to www.telekom.com and
% mention the original authors and Deutsche Telekom AG in a suitable manner. In
% a paper-like reuse please insert a short hint to www.telekom.com and to the
% original authors and Deutsche Telekom AG into your preface. For normal
% quotations please use the scientific standard to cite.
%
% [ File structure derived from 'mind your Scholar Research Framework' 
%   mycsrf (c) K. Reincke CC BY 3.0  http://mycsrf.fodina.de/ ]

% the first time cite with all data, later with shorttitle
\jurabibsetup{citefull=first}

%%% (1) author / editor list configuration
%\jurabibsetup{authorformat=and} % uses 'und' instead of 'u.'
% therefore define your own abbreviated conjunction: 
% an 'and before last author explicetly written conjunction

% for authors in citations
\renewcommand*{\jbbtasep}{ u. } % bta = between two authors sep
\renewcommand*{\jbbfsasep}{, } % bfsa = between first and second author sep
\renewcommand*{\jbbstasep}{ u. }% bsta = between second and third author sep
% for editors in citations
\renewcommand*{\jbbtesep}{ u. } % bta = between two authors sep
\renewcommand*{\jbbfsesep}{, } % bfsa = between first and second author sep
\renewcommand*{\jbbstesep}{ u. }% bsta = between second and third author sep

% for authors in literature list
\renewcommand*{\bibbtasep}{ u. } % bta = between two authors sep
\renewcommand*{\bibbfsasep}{, } % bfsa = between first and second author sep
\renewcommand*{\bibbstasep}{ u. }% bsta = between second and third author sep
% for editors  in literature list
\renewcommand*{\bibbtesep}{ u. } % bte = between two editors sep
\renewcommand*{\bibbfsesep}{, } % bfse = between first and second editor sep
\renewcommand*{\bibbstesep}{ u. }% bste = between second and third editor sep

% use: name, forname, forname lastname u. forname lastname
\jurabibsetup{authorformat=firstnotreversed}
\jurabibsetup{authorformat=italic}

%%% (2) title configuration
% in every case print the title, let it be seperated from the 
% author by a colon and use the slanted font
\jurabibsetup{titleformat={all,colonsep}}
%\renewcommand*{\jbtitlefont}{\textit}

%%% (3) seperators in bib data
% separate bibliographical hints and page hints by a comma
\jurabibsetup{commabeforerest}

%%% (4) specific configuration of bibdata in quotes / footnote
% use a.a.O if possible
\jurabibsetup{ibidem=strict}

% replace ugly a.a.O. by ders., a.a.O. resp. ders., ebda.
% but if there are more than one author or girl writers?
\AddTo\bibsgerman{
  \renewcommand*{\ibidemname}{Ds., a.a.O.}
  \renewcommand*{\ibidemmidname}{ds., a.a.O.}
}
\renewcommand*{\samepageibidemname}{Ds., ebda.}
\renewcommand*{\samepageibidemmidname}{ds., ebda.}

%%% (5) specific configuration of bibdata in bibliography
% ever an in: before journal and collection/book-tiltes 
\renewcommand*{\bibbtsep}{in: }
%\renewcommand*{\bibjtsep}{in: }

% ever a colon after author names 
\renewcommand*{\bibansep}{: }
% ever a semi colon after the title 
\renewcommand*{\bibatsep}{; }
% ever a comma before date/year
\renewcommand*{\bibbdsep}{, }

% let jurabib insert the S. and p. information
% no S. necessary in bib-files and in cites/footcites
\jurabibsetup{pages=format}

% use a compressed literature-list using a small line indent
\jurabibsetup{bibformat=compress}
\setlength{\jbbibhang}{1em}

% which follows the design of the cites and offers comments
\jurabibsetup{biblikecite}

% print annotations into bibliography
\jurabibsetup{annote}
\renewcommand*{\jbannoteformat}[1]{{ \itshape #1 }}

%refine the prefix of url download
\AddTo\bibsgerman{\renewcommand*{\urldatecomment}{Referenzdownload: }}

% we want to have the year of articles in brackets
\renewcommand*{\bibaldelim}{(}
\renewcommand*{\bibardelim}{)}

%Umformatierung des Reihentitels und der Reihennummer
\DeclareRobustCommand{\numberandseries}[2]{%
\unskip\unskip%,
\space\bibsnfont{(=~#2}%
\ifthenelse{\equal{#1}{}}{)}{, [Bd./Nr.]~#1)}%
}%


% language specific hyphenation
% Telekom osCompendium osHyphenation Include Module
%
% (c) Karsten Reincke, Deutsche Telekom AG, Darmstadt 2011
%
% This LaTeX-File is licensed under the Creative Commons Attribution-ShareAlike
% 3.0 Germany License (http://creativecommons.org/licenses/by-sa/3.0/de/): Feel
% free 'to share (to copy, distribute and transmit)' or 'to remix (to adapt)'
% it, if you '... distribute the resulting work under the same or similar
% license to this one' and if you respect how 'you must attribute the work in
% the manner specified by the author ...':
%
% In an internet based reuse please link the reused parts to www.telekom.com and
% mention the original authors and Deutsche Telekom AG in a suitable manner. In
% a paper-like reuse please insert a short hint to www.telekom.com and to the
% original authors and Deutsche Telekom AG into your preface. For normal
% quotations please use the scientific standard to cite.
%
% [ File structure derived from 'mind your Scholar Research Framework' 
%   mycsrf (c) K. Reincke CC BY 3.0  http://mycsrf.fodina.de/ ]
%


\hyphenation{rein-cke}




%%% (3) layout page configuration %%%

% select the visible parts of a page
% S.31: { plain|empty|headings|myheadings }
%\pagestyle{myheadings}
\pagestyle{headings}

% select the wished style of page-numbering
% S.32: { arabic,roman,Roman,alph,Alph }
\pagenumbering{arabic}
\setcounter{page}{1}

% select the wished distances using the general setlength order:
% S.34 { baselineskip| parskip | parindent }
% - general no indent for paragraphs
\setlength{\parindent}{0pt}
\setlength{\parskip}{1.2ex plus 0.2ex minus 0.2ex}


%%% (4) general package activation %%%
%\usepackage{utopia}
%\usepackage{courier}
%\usepackage{avant}
\usepackage[dvips]{epsfig}

% graphic
\usepackage{graphicx,color}
\usepackage{array}
\usepackage{shadow}
\usepackage{fancybox}

%- start(footnote-configuration)
%  flush the cite numbers out of the vertical line and let
%  the footnote text directly start in the left vertical line
\usepackage[marginal]{footmisc}
%- end(footnote-configuration)

\usepackage{amssymb}
\begin{document}

%% use all entries of the bliography

%%-- start(titlepage)
\titlehead{Literaturexzerpt}
\subject{Autor(en): Brügger, Harhoff. Picot, Creighton, Fiedler, Henkel}
\title{Titel: Open-Source-Software. Eine ökonomische und technische Analyse}
\subtitle{Jahr: 2004 }
\author{K. Reincke% Telekom osCompendium License Include Module
%
% (c) Karsten Reincke, Deutsche Telekom AG, Darmstadt 2011
%
% This LaTeX-File is licensed under the Creative Commons Attribution-ShareAlike
% 3.0 Germany License (http://creativecommons.org/licenses/by-sa/3.0/de/): Feel
% free 'to share (to copy, distribute and transmit)' or 'to remix (to adapt)'
% it, if you '... distribute the resulting work under the same or similar
% license to this one' and if you respect how 'you must attribute the work in
% the manner specified by the author ...':
%
% In an internet based reuse please link the reused parts to www.telekom.com and
% mention the original authors and Deutsche Telekom AG in a suitable manner. In
% a paper-like reuse please insert a short hint to www.telekom.com and to the
% original authors and Deutsche Telekom AG into your preface. For normal
% quotations please use the scientific standard to cite.
%
% [ File structure derived from 'mind your Scholar Research Framework' 
%   mycsrf (c) K. Reincke CC BY 3.0  http://mycsrf.fodina.de/ ]
%
\footnote{
This text is licensed under the Creative Commons Attribution-ShareAlike 3.0 Germany
License (http://creativecommons.org/licenses/by-sa/3.0/de/): Feel free \enquote{to
share (to copy, distribute and transmit)} or \enquote{to remix (to
adapt)} it, if you \enquote{[\ldots] distribute the resulting work under the
same or similar license to this one} and if you respect how \enquote{you
must attribute the work in the manner specified by the author(s)
[\ldots]}):
\newline
In an internet based reuse please mention the initial authors in a suitable
manner, name their sponsor \textit{Deutsche Telekom AG} and link it to
\texttt{http://www.telekom.com}. In a paper-like reuse please insert a short
hint to \texttt{http://www.telekom.com}, to the initial authors, and to their
sponsor \textit{Deutsche Telekom AG} into your preface. For normal citations
please use the scientific standard.
\newline
{ \tiny \itshape [derived from myCsrf (= 'mind your Scholar Research Framework') 
\copyright K. Reincke CC BY 3.0  http://mycsrf.fodina.de/)] }}}

%\thanks{den Autoren von KOMA-Script und denen von Jurabib}
\maketitle
%%-- end(titlepage)
%\nocite{*}

\begin{abstract}
\noindent
Das Werk / The work\footcite[][]{BruHarPicCreFieHen2004a} \\
\noindent \itshape
\ldots Möchte eigenem Bekunden nach 'mit den wesentlichen Einflussfaktoren' von
OSS vertraut machen: es clustert die Lizenzen und liefert neben Beispielen und
Aussagen zum Entwicklungsprozess auch Aussagen über die Beteiligung von Firmen
an der Nutzung und Entwicklung von Open Source Software. Interessanterweise wird
aber die Pflicht, die Firmen mit der Nutzung eingehen, nicht wirklich
thematisiert. \\
\noindent
\ldots This book wants to explain central factors of OSS: it lists examples,
clusters OS licenses, describes the development process and analyzes the OSS
work of companies. But nevertheless it does not describe in a real sense what
companies have to do to fulfill the licenses. For the authors these obligations
seem to be clear.
\end{abstract}
\footnotesize
%\tableofcontents
\normalsize

\section{Line of Thought}

Zuerst [Kap2] wird eine Analyse des Marktes vorgestellt, die heute, ca 6-7 Jahre
später veraltet sein sollte \footcite[vgl.][7-18]{BruHarPicCreFieHen2004a}

Sodann [Kap3] werden die OSS-Lizenzen katalogisiert und klassifiziert und
entsprechende Software beispielhaft
besprochen\footcite[vgl.][19-62]{BruHarPicCreFieHen2004a}. Dabei wird die GPL
als die relevante Lizenz schlechthin über eine Analyse der in
SourceForge verwendeten Lizenzen
ermittelt\footcite[vgl.][24 Dieses Verfahren dürfte
heute eher falsche Zahlen liefern, dennn Ecllipse und
Apache werden mittlerweil woanders gehostet. Stimmt
das?]{BruHarPicCreFieHen2004a}

Danach [Kap4] werden \enquote{technische Aspekte der Entwicklung und des Einsatzes
von OSS} besprochen\footcite[vgl.][63-94]{BruHarPicCreFieHen2004a},
insbesondere auch solche des
\enquote{Software-Ent\-wick\-lungs\-pro\-zes\-ses}\footcite[vgl.][67ff]{BruHarPicCreFieHen2004a}.

Zentral [Kap5] ist dann auch die Darstellung \enquote{ökonomischer Aspekte
der Entwicklung und des Einsatzes von
OSS}\footcite[vgl.][95-124]{BruHarPicCreFieHen2004a}, weil hier nicht nur
die Ökonomie der OS selbst thematisiert wird, sondern insbesondere auch die
\enquote{Initiierung und Unterstützung von OSS-Projekten durch
Unternehmen}\footcite[vgl.][101-114]{BruHarPicCreFieHen2004a}

Schließlich werden [Kap6] allgemeine \enquote{Rahmenbedingungen} und
rechtliche Fragen für \enquote{OSS und proprietäre
Software}\footcite[vgl.][125-164]{BruHarPicCreFieHen2004a} und
\enquote{Auswirkungen von
OSS}\footcite[vgl.][165-176]{BruHarPicCreFieHen2004a} samit
\enquote{Ausblick}\footcite[vgl.][177-182]{BruHarPicCreFieHen2004a}
diskutiert.

In unserem Zusammenhang fällt auf, dass man im Rahmen der
Lizenzanalyse\footcite[vgl.][19-62]{BruHarPicCreFieHen2004a} und insbesondere
bei der Betrachtung \enquote{unternehmerischer
Unterstützung}\footcite[vgl.][101-114]{BruHarPicCreFieHen2004a} hätte
erwarten können, dass die praktischen Pflichten der Unternehmen näher erläutert
werden, die sie mit der Nutzung, der Distribution oder Modifikation von OSS
eingehen. Richtig ist, dass es in beiden Abschnitten Ansätze dazu gibt: 

So enthält einerseits die OS Lizenzanalyse\footcite[was als Open Source Lizenz
gilt wird analog der OSI/OSDL festgelegt. Vgl.
dazu][20]{BruHarPicCreFieHen2004a} eine Klassifikation der Lizenzen nach
folgendem Muster\footcite[vgl.][23 - die Klassifikation in Form einer Tabelle
erfolgt explizit \enquote{in Anlehnung an Perens (1999)} und ist mit dem
Label \enquote{Eigenschaften von OSS-Lizenzen}
untertitelt]{BruHarPicCreFieHen2004a}:

\begin{tabular}[h]{|r|c|c|c|c|}
& \rotatebox{90}{\enquote{Quellcode  kann unbegrenzt gelesen, genutzt, modifiziert
und distribuiert werden}} 
& \rotatebox{90}{\enquote{Kann mit proprietärer Software verbunden und
(re-)distribuiert werden ohne OSS-Lizenz}} 
& \rotatebox{90}{\enquote{Modifikationen am OSS lizenzierten Quellcode können im
Distributionsfall proprietär bleiben}} 
& \rotatebox{90}{\enquote{Spezielle Privilegien für den ursprünglichen Copyright
halter über Modifikationen anderer}} 
\\
\hline \hline
GPL & \checkmark & & & \\
\hline
LGPL & \checkmark & \checkmark & &\\
\hline
BSD-Typ (MIT, Apache, W3c, Python, Zope) & \checkmark & \checkmark & \checkmark&\\
\hline
Artistic & \checkmark & \checkmark & \checkmark &\\
\hline
MPL & \checkmark & \checkmark &  & \checkmark\\
\hline
\end{tabular}

Oder es enthält etwa die Analyse der \enquote{ökonomischen Aspekte} eine
Typologisierung hinsichtlich er \enquote{[\ldots] Roller, die das Unternehmen in
Bezug auf die betreffende Software
einnimmt}\footcite[vgl.][102]{BruHarPicCreFieHen2004a}, nämlich (a) die
\enquote{Bereitstellung von OSS als Komplement zu eigenen Produkten}, (b)
die \enquote{Nutzung von OSS in internen Prozessen},(c) die
\enquote{Verwendung von OSS in eigenen Pprodukten} und (d) die
\enquote{Verwendung von OSS als
Kerngeschäft}\footcite[vgl.][103]{BruHarPicCreFieHen2004a}. 

Und mehr noch,
es wird sogar explizit der Fall von 'LinkSys' beschrieben, die als
\enquote{Routerhersteller für kleine Netzwerke} in ihren Geräten GPL
lizenzierten Code verwendet hätten und dann von der \enquote{Free Software
Foundation} aufgefordert worden seien, \enquote{[\ldots] den
Quellcode verfügbar zu
machen}\footcite[vgl.][111]{BruHarPicCreFieHen2004a}

Ja schließlich werden sogar mögliche Vor- und Nachteile der Open Source
Nutzung, Distribution oder Modifikation gegenübergestellt. Als Vorteile gelten
etwa\footcite[vgl.][114]{BruHarPicCreFieHen2004a}:
\begin{itemize}
  \item \enquote{externe Entwicklungsunterstützung}
  \item \enquote{reduzierte Kosten der Maintenance}
  \item \enquote{Standardsetzung}
  \item \enquote{Reputation[s]gewinne (bei gutem Code)}
  \item \enquote{erhöhte Programmierdisziplin}
  \item \enquote{Ruf als 'guter OSS-Player'}
  \item \enquote{erhöhte Attraktivität als Arbeitgeber für Programmierer}
  \item \enquote{erhöhte Nachfrage nach Komplement / Produkt}
  \item \enquote{Druck auf Anbieter proprietärer Software}
  \item \enquote{reduzierte Abhängigkeit von proprietärer Software}
  \item \enquote{offener Umgang mit Sicherheitslücken}
  \item \enquote{Rückzug aus Unterstützung alter Software möglich}
\end{itemize}

Und als Nachteile gelten dann\footcite[vgl.][114]{BruHarPicCreFieHen2004a}:
\begin{itemize}
  \item \enquote{Risiko des Forkings}
  \item \enquote{Verlust von Wettbewerbsvorteilen}
  \item \enquote{Reputationsverlust (bei schlechtem Code)}
  \item \enquote{Kosten für Vorbereitung des des Codes}
  \item \enquote{Verletzung von Schutzrechten können leichter entdeckt
  werden}
  \item \enquote{Kosten der Maintenance eines OSS-Projektes}
  \item \enquote{OSS kommt mglw. in erster Linie Wettbewerbern zugute}
  \item \enquote{Verlust von Bündelungsvorteilen}
  \item \enquote{Einsicht für Wettbewerber in Geschäftsprozesse}
  \item \enquote{Sichtbarkeit von Sicherheitslücken}
  \item \enquote{Lizenzgebühren ausgeschlossen}
\end{itemize}

Und trotz dieses Gedankengangs und trotz dieser spezifischen Auflistung wird
das, was Unternehmen konkret tun müssen, wenn sie eine OSS-Lizenz erfüllen
wollen, in diesem Zusammenhang nicht erwähnt. Es ist hier - wie auch sonst so
oft -, als würde schlicht vorausgesetzt, dass sich die notwendigen Pflichten
implizit und einfach (sic!) aus den Lizenzen ergäben und deshalb keiner weiteren
Worte mehr bedürfte.

\section{Specific Aspects}

Für den beschriebenen Fall der Lizenzverletzung durch LinkSys, die so die
Autoren - in ihren Routerprodukten GPL lizenzierte Software verwendet hätten und
deshalb von der FSF zur Offenlegung ihres (modifizierten) Codes gezwungen worden
seien, konstieren die Autoren auch, dass es \enquote{unangebracht} sei,
\enquote{[\ldots] bei der Verwendung von GPL-Software in eingebetteten Systemen
(von einem 'Risiko') zu
sprechen}\footcite[vgl.][111]{BruHarPicCreFieHen2004a}. Denn \begin{quote}
\enquote{Das Problem liegt nicht in den Bedingungen der GPL, sondern im Versuch,
Software unter Missachtung der Lizenzbedingungen und somit widerrechtlich zu
nutzen.}\footcite[][111]{BruHarPicCreFieHen2004a}
\end{quote}
\small
\bibliography{../bibfiles/oscResourcesDe}

\end{document}
