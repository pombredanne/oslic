% Telekom osCompendium extract template
%
% (c) Karsten Reincke, Deutsche Telekom AG, Darmstadt 2011
%
% This LaTeX-File is licensed under the Creative Commons Attribution-ShareAlike
% 3.0 Germany License (http://creativecommons.org/licenses/by-sa/3.0/de/): Feel
% free 'to share (to copy, distribute and transmit)' or 'to remix (to adapt)'
% it, if you '... distribute the resulting work under the same or similar
% license to this one' and if you respect how 'you must attribute the work in
% the manner specified by the author ...':
%
% In an internet based reuse please link the reused parts to www.telekom.com and
% mention the original authors and Deutsche Telekom AG in a suitable manner. In
% a paper-like reuse please insert a short hint to www.telekom.com and to the
% original authors and Deutsche Telekom AG into your preface. For normal
% quotations please use the scientific standard to cite.
%
% [ File structure derived from 'mind your Scholar Research Framework' 
%   mycsrf (c) K. Reincke CC BY 3.0  http://mycsrf.fodina.de/ ]

%
% select the document class
% S.26: [ 10pt|11pt|12pt onecolumn|twocolumn oneside|twoside notitlepage|titlepage final|draft
%         leqno fleqn openbib a4paper|a5paper|b5paper|letterpaper|legalpaper|executivepaper openrigth ]
% S.25: { article|report|book|letter ... }
%
% oder koma-skript S.10 + 16
\documentclass[DIV=calc,BCOR=5mm,11pt,headings=small,oneside,abstract=true, toc=bib]{scrartcl}

%%% (1) general configurations %%%
\usepackage[utf8]{inputenc}

%%% (2) language specific configurations %%%
\usepackage[]{a4,ngerman}
\usepackage[ngerman, english]{babel}
\selectlanguage{english}

%language specific quoting signs
%default for language emglish is american style of quotes
\usepackage{csquotes}

% jurabib configuration
\usepackage[see]{jurabib}
\bibliographystyle{jurabib}
\input{../btexmat/oscJbibCfgEnInc}

% language specific hyphenation
% Telekom osCompendium osHyphenation Include Module
%
% (c) Karsten Reincke, Deutsche Telekom AG, Darmstadt 2011
%
% This LaTeX-File is licensed under the Creative Commons Attribution-ShareAlike
% 3.0 Germany License (http://creativecommons.org/licenses/by-sa/3.0/de/): Feel
% free 'to share (to copy, distribute and transmit)' or 'to remix (to adapt)'
% it, if you '... distribute the resulting work under the same or similar
% license to this one' and if you respect how 'you must attribute the work in
% the manner specified by the author ...':
%
% In an internet based reuse please link the reused parts to www.telekom.com and
% mention the original authors and Deutsche Telekom AG in a suitable manner. In
% a paper-like reuse please insert a short hint to www.telekom.com and to the
% original authors and Deutsche Telekom AG into your preface. For normal
% quotations please use the scientific standard to cite.
%
% [ File structure derived from 'mind your Scholar Research Framework' 
%   mycsrf (c) K. Reincke CC BY 3.0  http://mycsrf.fodina.de/ ]
%


\hyphenation{rein-cke}
\hyphenation{Rein-cke}
\hyphenation{OS-LiC}
\hyphenation{ori-gi-nal}
\hyphenation{bi-na-ry}
\hyphenation{Li-cen-ce}
\hyphenation{li-cen-ce}


%%% (3) layout page configuration %%%

% select the visible parts of a page
% S.31: { plain|empty|headings|myheadings }
%\pagestyle{myheadings}
\pagestyle{headings}

% select the wished style of page-numbering
% S.32: { arabic,roman,Roman,alph,Alph }
\pagenumbering{arabic}
\setcounter{page}{1}

% select the wished distances using the general setlength order:
% S.34 { baselineskip| parskip | parindent }
% - general no indent for paragraphs
\setlength{\parindent}{0pt}
\setlength{\parskip}{1.2ex plus 0.2ex minus 0.2ex}


%%% (4) general package activation %%%
%\usepackage{utopia}
%\usepackage{courier}
%\usepackage{avant}
\usepackage[dvips]{epsfig}

% graphic
\usepackage{graphicx,color}
\usepackage{array}
\usepackage{shadow}
\usepackage{fancybox}

%- start(footnote-configuration)
%  flush the cite numbers out of the vertical line and let
%  the footnote text directly start in the left vertical line
\usepackage[marginal]{footmisc}
%- end(footnote-configuration)

\begin{document}

%% use all entries of the bliography

%%-- start(titlepage)
\titlehead{Literaturexzerpt}
\subject{Autor(en): Williams / Williams2002a}
\title{Titel: Free as in Freedom: Richard Stallman's Crusade\ldots}
\subtitle{Jahr: \ldots }
\author{K. Reincke% Telekom osCompendium License Include Module
%
% (c) Karsten Reincke, Deutsche Telekom AG, Darmstadt 2011
%
% This LaTeX-File is licensed under the Creative Commons Attribution-ShareAlike
% 3.0 Germany License (http://creativecommons.org/licenses/by-sa/3.0/de/): Feel
% free 'to share (to copy, distribute and transmit)' or 'to remix (to adapt)'
% it, if you '... distribute the resulting work under the same or similar
% license to this one' and if you respect how 'you must attribute the work in
% the manner specified by the author ...':
%
% In an internet based reuse please link the reused parts to www.telekom.com and
% mention the original authors and Deutsche Telekom AG in a suitable manner. In
% a paper-like reuse please insert a short hint to www.telekom.com and to the
% original authors and Deutsche Telekom AG into your preface. For normal
% quotations please use the scientific standard to cite.
%
% [ File structure derived from 'mind your Scholar Research Framework' 
%   mycsrf (c) K. Reincke CC BY 3.0  http://mycsrf.fodina.de/ ]
%
\footnote{
This text is licensed under the Creative Commons Attribution-ShareAlike 3.0 Germany
License (http://creativecommons.org/licenses/by-sa/3.0/de/): Feel free \enquote{to
share (to copy, distribute and transmit)} or \enquote{to remix (to
adapt)} it, if you \enquote{[\ldots] distribute the resulting work under the
same or similar license to this one} and if you respect how \enquote{you
must attribute the work in the manner specified by the author(s)
[\ldots]}):
\newline
In an internet based reuse please mention the initial authors in a suitable
manner, name their sponsor \textit{Deutsche Telekom AG} and link it to
\texttt{http://www.telekom.com}. In a paper-like reuse please insert a short
hint to \texttt{http://www.telekom.com}, to the initial authors, and to their
sponsor \textit{Deutsche Telekom AG} into your preface. For normal citations
please use the scientific standard.
\newline
{ \tiny \itshape [derived from myCsrf (= 'mind your Scholar Research Framework') 
\copyright K. Reincke CC BY 3.0  http://mycsrf.fodina.de/)] }}}

%\thanks{den Autoren von KOMA-Script und denen von Jurabib}
\maketitle
%%-- end(titlepage)
%\nocite{*}

\begin{abstract}
Das Werk / The work\footcite[][]{Williams2002a} \\
\noindent \itshape
\ldots Die erste(?) Biographie über RMS: sie enthält so viele Interviews and
Hintergrundinformationen, dass sie als Primärquelle genommen werden kann, und
zwar nicht nur für die Geschichte von RMS selbst, sondern auch für die
Geschichte der 'Freien Software' und ihrer internen Gegenspielerin 'Open Source'. \\
\noindent
\ldots This is the first(?) biography of RMS. It contains so many interviews and
background information that it might be read as primary source, not only for the
history of RMS, but also for the history of the Free Software movement and its
internal counterpart, the Open Source movement.
\end{abstract}
\footnotesize
%\tableofcontents
\normalsize

\section{Line of Thought}

\subsection{The Printer as Start}

Williams describes RMS's initial moment of realizing that closed software has
victims with all the background knowledge:

\begin{itemize}
  \item First, one has to note that in the 1960's and 1970's there was a
  \enquote{give-and-take philosophy} in the MIT AI lab which includes not
  only the developer but also the computer companies. And \enquote{such a
  philosophy was a major reason why companies [\ldots] made it a policy
  to donate their machines and software programs where hackers typically
  congregated}. The hackers improved the software and \enquote{[\ldots]
  companies could borrow back the improvements [\ldots]}: these hackers
  were \enquote{[\ldots] an auxiliary research-and-development division
  available at minimal cost}\footcite[cf][4]{Williams2002a}. In this
  environment even RMS had already debugged printer software
  successfully\footcite[cf][3 and 5]{Williams2002a}.
  \item But then a new Xerox laser printer came, \enquote{a gift to good to
  refuse}\footcite[cf][2]{Williams2002a}, by which the situation was
  changed: \enquote{[\ldots] Xerox had been unwilling to share its
  sourcecode files} and Stallman didn't get the sourcecode by Xerox even
  in a situation where this new printer had also (old)
  problems\footcite[cf][6]{Williams2002a}. Stallmans next step was to contact a
  programming colleague at Carnegie mellon AI lab\footcite[cf][7]{Williams2002a}
  for getting the code from one peer to another\footcite[cf][8]{Williams2002a}.
  But there RMS got the famous answer, that his colleague \enquote{[\ldots]
  had promised not to give (him, RMS) a
  copy}\footcite[cf][8]{Williams2002a} - that answer, which evoked - in
  the words of RMS - that he noted that non-disclosure answers have
  victims\footcite[cf][158]{Stallman2001a} - and which made him - following the
  narration of Williams - \enquote{[\ldots] so angry (he) couldn't think of a way
  to express it}\footcite[cf][9]{Williams2002a}. Because of these emotions
  of being helpless and at the mercy of another - so Williams - RMS later
  decided to help himself and others by avoiding such situations and writing the
  free operating system and fighting for free
  software\footcite[cf][11f]{Williams2002a}.
  \item But even inside the AI lab there was a movement to establish more
  secrets, for example by introducing passwords. RMS fought also against this
  movement, firstly by using an empty password, later on by requiring others
  also to use an empty password because it were \enquote{[\ldots] much easier to
  type} and finally by using his account 'rms' as password too and by
  making this fact known to other programmers, even outside of the
  lab\footcite[cf][53ff, 93f et passim]{Williams2002a}
\end{itemize}

\subsection{The Young RMS}

Williams suggests by quoting some witnesses that RMS might \enquote{[\ldots]
have had some of the qualities of an autistic child}, features, which
later on lead to his radicalism, unwillingness to make a comprize and his and
obstinacy\footcite[cf][31 et passim]{Williams2002a}.

\subsection{The Emacs Commune}

Following Williams \enquote{the story of Stallman's work on TECO during 1970's is
inextricably linked with Stallman's later leadershipd of the free software
movement}\footcite[cf][80]{Williams2002a}. That was concluded by the
following line of thoughts: First, TECO was an editor which required that the
user inserted \enquote{extended series of editing instructions} before he
could see the modified text on the screen\footcite[cf][81]{Williams2002a}. Later
on this editor was improved by the possiblity to (re)call such a set of
instructions by a simplier key stroke, hence by a
macro\footcite[cf][82]{Williams2002a}. Based on this feature finally Stallman
published a set of macros und the project name 'editing macros' or abbreviated
as \enquote{emacs}\footcite[cf][84f]{Williams2002a}. The philosophy of these
improvements was, that \enquote{users were free to modify and redistribute the
code on the condition that they gave back all the extensions they
made}\footcite[cf][85]{Williams2002a}

\subsection{The Changings (moral choice) 7}

Following Williams RMS announced \enquote{on September 27, 1983} the
following text\footcite[cf][85]{Williams2002a}:

\begin{quote}
\enquote{Starting this Thanksgiving I [RMS; KR] am going to write a complete
Unix-compatible software system called GNU (for GNU's Not Unix), and give it away free to
everyone who can use it.
}\footcite[][85]{Williams2002a}
\end{quote}

As Williams notes \enquote{Stallman's decision to start developing the GNU
system was triggered by the end of the ITS system that the AI Lab hackers
had nurtured for so long}\footcite[][91]{Williams2002a}. With the decision
of the \enquote{lab's administration} in 1982 \enquote{[\ldots] to
upgrade its main computer} the AI lab decided also, to switch to
\enquote{a commercial operating system developed by Digital}. And this
decision was based upon the fear that \enquote{[\ldots] the lab had lost
its critical mass of in-house programming
talent}\footcite[cf][92]{Williams2002a}. Hence Stallman met proprietary
closed software\footcite[cf][92]{Williams2002a}

A second a movement which might have supported the decision to write his own
operating system were the experiences Stallman made with the 'lisp machine':
Stallman had had the \enquote{gentleman agreement} with Symbolics to
\enquote{review} the source code\footcite[cf][95]{Williams2002a}. With the
revocation of their agreement on March 16,1982
the so called \enquote{Symbolics War of 1982-1983}
started\footcite[cf][96]{Williams2002a} in which RMS tried to reimplement all
new features of the new lisp machine into the older state only by looking at
the IO of the new program: RMS tried obtain the freedom of the AI lab by redoing
the complete job, the company had already done. And he \enquote{[\ldots] made
sure LMI programmers had direct access to the
changes}\footcite[cf][97]{Williams2002a}. At the end the company won,
stallman was seen \enquote{as a troubling anachronism} which did not see the
advantages \enquote{in commercializing the Lisp
Machine}\footcite[cf][97]{Williams2002a}.

As consequence RMS decided to write his own free operating system, called
GNU\footcite[cf][101]{Williams2002a}: he faced a \enquote{stark moral
choice} - in the word of Williams - he had \enquote{[\ldots] either
(to) get over his ethical objection for 'proprietary' software [\ldots]
or (to) dedicate his life to building an alternate, nonproprietary
system of software programs}\footcite[cf][101]{Williams2002a}.

Williams now tells the story as it is known: 
\begin{itemize}
   \item in Januray 1984 RMS \enquote{[\ldots]
  resigned from the MIT staff [\ldots] to build GNU}. But he had
  \enquote{enough friends and allies within the AI Lab to retain rent-free
  access to his MIT office}. The value of the resigning was to
  \enquote{[\ldots] (negate) any debate about conflict of interists or
  Institute ownership of the software}\footcite[cf][102]{Williams2002a}
  \item One of the first development task was (re)writing
  Emacs\footcite[cf][105]{Williams2002a}
  \item In 1985 with the GNU Manifesto the political ethical embeddation
  started\footcite[cf][105]{Williams2002a} which was expanded by the foundation
  of the Free Software Foundation\footcite[cf][106]{Williams2002a} 
\end{itemize}

\subsection{GPL}

\subsection{GNU/Emacs}

The beginning of the GPL starts with publishing and distributing the
GNU/Emacs\footcite[cf][123]{Williams2002a}, but in a stronger version as it is
known today: In these days Stallman allows the users \enquote{[\ldots] to
modify GNU Emacs just so long as they published their
modifications}\footcite[cf][124]{Williams2002a}. This was the obligation
to publish your modifications regardless wether you handover your modifications
to others or not. Later on Stallman only revoked this obligation because of the
\enquote{Big Brother aspect of the original Emacs Commune social contract}.
Stallman shall have said that \enquote{it was wrong to require to publish all
changes}\footcite[cf][127]{Williams2002a}.

And therefore the GPL only requires to handover the modified code to those  to
whom you distribute the modified  binary\footcite[cf][128ff]{Williams2002a}.

\subsection{GNU/LInux}

The Open Source Story in this view starts with the lack of a kernel in the GNU
operating system which originally should be closed by a GNU kernel, named
\enquote{HURD}. But one has to state, that \enquote{by 1993, the GNU Project's
inability to deliver a working kernel was leading to problems both within the
GNU Project and within the free software movement at
large}\footcite[cf][145]{Williams2002a}. On the other hand the Linux
kernel began to become reality in these years
and his developers indeed uses many GNU
tools for developing it and for making it part of
a meaningful environment: a pure kernel without an
operating system is nothing\footcite[cf][143]{Williams2002a}.

The next step of the history of a whole operating system is the Debian
distribution and the Debian manifesto\footcite[cf][147]{Williams2002a} and so in
the later 1980s Stallman fought to name the complete system GNU/Linux instead of
only naming it 'Linux'\footcite[cf][149f]{Williams2002a}

\subsection{Open Source}

The step into Open Source began with the new observation, that developing the
Linux kernel had used a new developing method, described in the article
\enquote{The Cathedral and the Bazaar}, written by Eric S.
Raymond\footcite[cf][159]{Williams2002a}, He had observed the methods of Linus
Torvalds and tranferred them to his own project
'fetchmail'\footcite[cf][158]{Williams2002a} and summarized his expericenses in
this article, presented at some conferences\footcite[cf][159]{Williams2002a}

This new development method seduced Netscape to come closer to the free software
movement: the wanted to publish their browser\footcite[cf][161]{Williams2002a}.

And in one of these discussions between Raymond and Netscape one wanted to state
that \enquote{despite the best efforts of Stallman and other hackers to
remind people that the word 'free' in free software stood for freedom
[SW] and not price, the message still wan't getting
throught}\footcite[cf][161f]{Williams2002a}

\begin{quote}
\enquote{Most business executives, upon hearing the term for the first time,
interpreted the word as synonymous with 'zero cost' [\ldots]
}\footcite[][162]{Williams2002a}
\end{quote}

As result of these misfeelings  [Mrs?] Peterson invented the new term
\enquote{Open Source}\footcite[cf][162]{Williams2002a} and as concequence of
a summit,sponsored by O'Reilley - and without invitation andf participation of
RMS - \enquote{[\ldots] the term 'open source' won over just enough
summit-goers to qualify as a success}\footcite[cf][162f]{Williams2002a}

Stallmans reaction to this act of 'renaming' an idea \enquote{[\ldots] was
slow in coming}: only \enquote{by the end of 1998, Stallman had
formulated a position: open source, while helpful in communicating the
technical advantages of free software, also encouraged speakers to
soft-pedal the issue of software freedom. Given this drawback, Stallman
would stick withn the term free
software}\footcite[cf][165]{Williams2002a}
\section{Specific Aspects}

\subsection{viral nature}

Here you find a hint, that the 'viral' nature of the GPL has been published by
Craig Mundie added by the hint, that the GPL itself shall 'poses a threat'
\enquote{[\ldots] to any company that relieser on the uniqueness of its software
as a competitive asset}\footcite[cf][16]{Williams2002a}. And following
Williams, Mundie shall have added: \enquote{It also fundamentally undermines the
independent commercial software sector because it effectively makes it
impossible to distribute on a basis where recipients pay for the product rather
than just the cost of
distribution}\footcite[cf][16]{Williams2002a}. [Williams proves his
quote by claiming that he made an online transcript from speech 'The Commercial
Software Model' by Craig Mundie\footcite[cf][24]{Williams2002a}. And finally he
links this quote to the url:
http://www.microsoft.com/presspass/exec/craig/05-03sharedsource.asp]

\subsection{the beginning is proprietarization}

Williams mentions Bill Gates \enquote{Open Letter to Hobbyists} by which B.
Gates tried to establich the value of closed software programs by arguing on the
quality\footcite[cf][100]{Williams2002a}
\small
\bibliography{../bibfiles/oscResourcesEn}

\end{document}
