% Telekom osCompendium extract template
%
% (c) Karsten Reincke, Deutsche Telekom AG, Darmstadt 2011
%
% This LaTeX-File is licensed under the Creative Commons Attribution-ShareAlike
% 3.0 Germany License (http://creativecommons.org/licenses/by-sa/3.0/de/): Feel
% free 'to share (to copy, distribute and transmit)' or 'to remix (to adapt)'
% it, if you '... distribute the resulting work under the same or similar
% license to this one' and if you respect how 'you must attribute the work in
% the manner specified by the author ...':
%
% In an internet based reuse please link the reused parts to www.telekom.com and
% mention the original authors and Deutsche Telekom AG in a suitable manner. In
% a paper-like reuse please insert a short hint to www.telekom.com and to the
% original authors and Deutsche Telekom AG into your preface. For normal
% quotations please use the scientific standard to cite.
%
% [ File structure derived from 'mind your Scholar Research Framework' 
%   mycsrf (c) K. Reincke CC BY 3.0  http://mycsrf.fodina.de/ ]

%
% select the document class
% S.26: [ 10pt|11pt|12pt onecolumn|twocolumn oneside|twoside notitlepage|titlepage final|draft
%         leqno fleqn openbib a4paper|a5paper|b5paper|letterpaper|legalpaper|executivepaper openrigth ]
% S.25: { article|report|book|letter ... }
%
% oder koma-skript S.10 + 16
\documentclass[DIV=calc,BCOR=5mm,11pt,headings=small,oneside,abstract=true, toc=bib]{scrartcl}

%%% (1) general configurations %%%
\usepackage[utf8]{inputenc}

%%% (2) language specific configurations %%%
\usepackage[]{a4,ngerman}
\usepackage[ngerman, english]{babel}
\selectlanguage{english}

%language specific quoting signs
%default for language emglish is american style of quotes
\usepackage{csquotes}

% jurabib configuration
\usepackage[see]{jurabib}
\bibliographystyle{jurabib}
\input{../btexmat/oscJbibCfgEnInc}

% language specific hyphenation
% Telekom osCompendium osHyphenation Include Module
%
% (c) Karsten Reincke, Deutsche Telekom AG, Darmstadt 2011
%
% This LaTeX-File is licensed under the Creative Commons Attribution-ShareAlike
% 3.0 Germany License (http://creativecommons.org/licenses/by-sa/3.0/de/): Feel
% free 'to share (to copy, distribute and transmit)' or 'to remix (to adapt)'
% it, if you '... distribute the resulting work under the same or similar
% license to this one' and if you respect how 'you must attribute the work in
% the manner specified by the author ...':
%
% In an internet based reuse please link the reused parts to www.telekom.com and
% mention the original authors and Deutsche Telekom AG in a suitable manner. In
% a paper-like reuse please insert a short hint to www.telekom.com and to the
% original authors and Deutsche Telekom AG into your preface. For normal
% quotations please use the scientific standard to cite.
%
% [ File structure derived from 'mind your Scholar Research Framework' 
%   mycsrf (c) K. Reincke CC BY 3.0  http://mycsrf.fodina.de/ ]
%


\hyphenation{rein-cke}
\hyphenation{Rein-cke}
\hyphenation{OS-LiC}
\hyphenation{ori-gi-nal}
\hyphenation{bi-na-ry}
\hyphenation{Li-cen-ce}
\hyphenation{li-cen-ce}


%%% (3) layout page configuration %%%

% select the visible parts of a page
% S.31: { plain|empty|headings|myheadings }
%\pagestyle{myheadings}
\pagestyle{headings}

% select the wished style of page-numbering
% S.32: { arabic,roman,Roman,alph,Alph }
\pagenumbering{arabic}
\setcounter{page}{1}

% select the wished distances using the general setlength order:
% S.34 { baselineskip| parskip | parindent }
% - general no indent for paragraphs
\setlength{\parindent}{0pt}
\setlength{\parskip}{1.2ex plus 0.2ex minus 0.2ex}


%%% (4) general package activation %%%
%\usepackage{utopia}
%\usepackage{courier}
%\usepackage{avant}
\usepackage[dvips]{epsfig}

% graphic
\usepackage{graphicx,color}
\usepackage{array}
\usepackage{shadow}
\usepackage{fancybox}

%- start(footnote-configuration)
%  flush the cite numbers out of the vertical line and let
%  the footnote text directly start in the left vertical line
\usepackage[marginal]{footmisc}
%- end(footnote-configuration)

\begin{document}

%% use all entries of the bliography

%%-- start(titlepage)
\titlehead{Literaturexzerpt}
\subject{Autor(en): Käs}
\title{Titel: Rethinking industry practice. The emergence of openness \ldots}
\subtitle{Jahr: 2008}
\author{K. Reincke% Telekom osCompendium License Include Module
%
% (c) Karsten Reincke, Deutsche Telekom AG, Darmstadt 2011
%
% This LaTeX-File is licensed under the Creative Commons Attribution-ShareAlike
% 3.0 Germany License (http://creativecommons.org/licenses/by-sa/3.0/de/): Feel
% free 'to share (to copy, distribute and transmit)' or 'to remix (to adapt)'
% it, if you '... distribute the resulting work under the same or similar
% license to this one' and if you respect how 'you must attribute the work in
% the manner specified by the author ...':
%
% In an internet based reuse please link the reused parts to www.telekom.com and
% mention the original authors and Deutsche Telekom AG in a suitable manner. In
% a paper-like reuse please insert a short hint to www.telekom.com and to the
% original authors and Deutsche Telekom AG into your preface. For normal
% quotations please use the scientific standard to cite.
%
% [ File structure derived from 'mind your Scholar Research Framework' 
%   mycsrf (c) K. Reincke CC BY 3.0  http://mycsrf.fodina.de/ ]
%
\footnote{
This text is licensed under the Creative Commons Attribution-ShareAlike 3.0 Germany
License (http://creativecommons.org/licenses/by-sa/3.0/de/): Feel free \enquote{to
share (to copy, distribute and transmit)} or \enquote{to remix (to
adapt)} it, if you \enquote{[\ldots] distribute the resulting work under the
same or similar license to this one} and if you respect how \enquote{you
must attribute the work in the manner specified by the author(s)
[\ldots]}):
\newline
In an internet based reuse please mention the initial authors in a suitable
manner, name their sponsor \textit{Deutsche Telekom AG} and link it to
\texttt{http://www.telekom.com}. In a paper-like reuse please insert a short
hint to \texttt{http://www.telekom.com}, to the initial authors, and to their
sponsor \textit{Deutsche Telekom AG} into your preface. For normal citations
please use the scientific standard.
\newline
{ \tiny \itshape [derived from myCsrf (= 'mind your Scholar Research Framework') 
\copyright K. Reincke CC BY 3.0  http://mycsrf.fodina.de/)] }}}

%\thanks{den Autoren von KOMA-Script und denen von Jurabib}
\maketitle
%%-- end(titlepage)
%\nocite{*}

\begin{abstract}
\noindent
as Werk / The work\footcite[][]{Kaes2008a} \\
\noindent \itshape
\ldots Untersucht den Wandel hin zur Offenheit in der Softwareentwicklung über
Interviews innerhalb von 'Embedded-Linux'-Firmen. Das Ergebnis ist, dass die
Offenheit kundengetrieben sei und zugleich einen Lernprozess innerhalb der
Firmen bedinge. Das Buch enthält eine kurze, aber präzise Darstellung dessen,
was OSS auch aus Lizenzsicht ist; eine systematische Darstellung der
eingegangenen Verpflichtungen liefert es nicht.\\
\noindent
\ldots This book analyzes the change to open software developement by
interviewing 'embedded-linux'-companies. The result is that openness is required by the
customers and that practicing openness evokes a process of learning. The book
offers a short but tellingly survey of OSS and their licences, although it
doesn't offer a systematical review of the obligations established by the
different licences.
\end{abstract}
\footnotesize
%\tableofcontents
\normalsize

Hint: Mss Käs uses 'licence' instead of 'license'. Be careful while quoting her.
\section{Line of Thought}

This book analyzes the change to open software developement by
interviewing 'embedded-linux'-companies, whereby these
\enquote{semi-structured interviews} were \enquote{[\ldots] conducted
by phone between November 2005 and November
2006}\footcite[cf][100f]{Kaes2008a} . 

The result is simple but important: there are \enquote{two keyfindings emerged
from the study}: At first it's the \enquote{key reason for component firms
to open up} that \enquote{[\ldots] customer[s] demand for
openness}\footcite[cf.][212]{Kaes2008a}. And secondly \enquote{[\ldots] the
change towards increased openness [/NextPage] requires substantial learning
effort by firms, which explains the incremental rather than sudden increase in
revealing}\footcite[cf.][212f]{Kaes2008a}

\section{Specific Aspects: Open Source Software [Chapter 3]}

\subsection{History}
Mss. Käs firstly outlines the history of the concept 'Open Source'. She
highlights that in the beginning there were a community exchanging software
freely. But - following Peerens (1999) \enquote{in the early 1980s software
vendors begang restricting the sharing of their software and charging
fees for each copy}. As reaction Mr. Stallmann had \enquote{[\ldots]
launched the GNU Project in 1984 to write a complete operating system
free from constraints on the use of its source code}. As core of the idea
Mss Käs says that \enquote{there are two preconditions to allow these
freedoms: copyright owner first have to provide access not only to the
machine-readable binary code, but also to the human-readable source coe,
and second have to remove legal restrictions from copyright
licences}\footcite[cf.][59]{Kaes2008a}. One instance of such a
\enquote{copyleft licence} were the GPL and \enquote{in 1991,
programmer Linus Torvalds [\ldots] started to work in the Linux kernel
under the GNU GPL}\footcite[cf.][60]{Kaes2008a}

\subsection{OSS Definition}
Finally Mss Käs explains the change away from the concept 'free software' to the
new concept 'open source software': free software as term were considered
\enquote{[\ldots] to hamper broad (and commercial) acceptance of the free
software model [\ldots]}, because \enquote{[\ldots]
the term 'free software' was often confused
with 'gratis' software [\ldots]}\footcite[cf.][60]{Kaes2008a}. Hence
\enquote{In 1998, the Open Source Initiative (OSI) was established to set
up a consistent set of criteria for what constituted an open source
licence, called the 'Open Source Definition'
(OSD)}\footcite[cf.][60]{Kaes2008a}. Naturally Mss Käs quotes the ten
criteria being known as OSD\footcite[cf.][62]{Kaes2008a}.

\subsection{important aspects: typology of licences}

First, Mss Käs underlines, \enquote{[\ldots] that the OSD also covers
non-copyleft licences (such as the BSD licence or the Apache licence) that
allow for software or modified versions of the software to be distributed as
proprietary software}\footcite[cf.][63]{Kaes2008a}

Second, she hints to the \enquote{copyleft (or viral) licences in the sense of
the FSF}: Following the explanations of Stallmann 2001 these copyleft
licences \enquote{[\ldots] require licensing back all 'derived work' from the
original software under an equivalent open source licence (reciprocity
obligation)}\footcite[cf.][63]{Kaes2008a}. And - so Mss Käs -
\enquote{it has to be noted [\ldots], that reciprocity obligations do not
arise with the use of (modified) software, but with its
\textit{distribution}}\footcite[cf.][63 emph.i.o.]{Kaes2008a}

[Comment: There is onyl ony weakness in this description: viral covers
recirpocity but not vice verse. The wish to get back the changes (resiprocity)
might be limited to the borders of the original and therefore might not be
concern the large work into which the original might be 'included'.]

Nevertheless Ms Käs also hints to Linus Torvalds (1999) [with out page] and the
fact, that \enquote{the defintion of \textit{derived work} under GPL is, however,
still vague [\ldots]} and that therefore Linus decided
that \enquote{[\ldots] system calls would not be considered to be linking against
the kernel}\footcite[cf.][63 emph.i.o. last part of the quote
is a requote of Linus statement quoted bei Käs]{Kaes2008a}

And finally Mss Käs hints also to the fact, that \enquote{the GPL does not
require a firm to give everybody access to the piece of
[modified; KR] software. It requires only that recipients of a copy of
the software [\ldots] be given the opportunity to receive the
sourcecode}\footcite[cf.][77]{Kaes2008a}. [Hint KR:] But to cover this
fact - as Mss Käs do - under the idea of \enquote{protection of a piece of
software}\footcite[cf.][77]{Kaes2008a} seems to be difficult. The GPL says
also, that the recipient gets the right to use, to modify and to distribute the
received code. Only in between different parts of a company or a poticila office
this interpretation could be a solution.

\subsection{the difference of work}

Mss Käs states:

\begin{quote} \enquote{Two aspects that make OSS development fundamentally
different from proprietary software development account for these
organizational and market value changes: release of control over the
source code and the opened-up, collaborative development
process.}\footcite[][65]{Kaes2008a}
\end{quote}
\small
\bibliography{../bibfiles/oscResourcesEn}

\end{document}
